\chapter{Поле 101: язык основного текста}

Поле обязательное.

Повторяемое. Повторяется для каждого языка, используемого в тексте.

Коды языков всегда вводятся строчными латинскими буквами, состоят из трех символов и берутся строго из встроенного справочника \emph{jz.mnu}!

Первым указывается язык, применяемый в данном документе в наибольшей степени. При неоднозначности (непонятно, какой из двух языков основной для документа) рекомендуется использовать код русского языка \emph{rus}, если, конечно, в документе имеется русский язык.

Небольшие цитаты (объёмом не более одного абзаца) не учитываются!

Для многоязычных документов (более 3 языков) можно использовать код языка \emph{mul} – "<мультиязычный документ">.

Для документов, язык которого определить невозможно (в том числе по причине отсутствия письменного текста и/или аудиосопровождения), нужно ис-пользовать код \emph{und} – <"не определено">.

Нельзя использовать вместо трехбуквенного кода языка двухбуквенный код страны.
