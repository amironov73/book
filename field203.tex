\chapter{Поле 203: вид содержания, средства доступа, характеристика содержания}

Поле введено в ИРБИС64+ версии 2018.1 в связи с переходом на ГОСТ 7.0.100-2018.

\section{Подполя A, B, D, E, F, G, I, K, L: вид содержания}

Значения берутся из справочника \emph{vso.mnu}.

\begin{cutelist}
    \item Движение
    \item Звуки
    \item Изображение
    \item Музыка
    \item Предмет
    \item Текст
    \item Устная речь
    \item Электронная программа
    \item Электронные данные
\end{cutelist}

\section{Подполя C, 1, 2, 3, 4, 5, 6, 7, 8: }

Значения берутся из справочника \emph{tso.mnu}.

\begin{cutelist}
    \item аудио
    \item видео
    \item микроскопический
    \item микроскопическое
    \item микроформа
    \item непосредственный
    \item непосредственное
    \item проекционный
    \item проекционное
    \item стереографический
    \item стереографическое
    \item электронный
    \item электронное
    \item электронная
    \item электронные
    \item разные средства доступа
    \item другое средство доступа
\end{cutelist}

\section{Подполя O, P, U, Y, T, R, W, Q, X, V, Z: характеристика содержания}

Значения берутся из справочника \emph{hso.mnu}.

\begin{cutelist}
    \item знаковая
    \item знаковое
    \item знаковый
    \item исполнительская
    \item исполнительское
    \item исполнительский
    \item картографическое
    \item картографический
    \item визуальная
    \item визуальное
    \item визуальный
    \item вкусовой
    \item обонятельный
    \item слуховая
    \item слуховой
    \item слуховые
    \item тактильная
    \item тактильное
    \item тактильный
    \item движущееся
    \item движущийся
    \item неподвижное
    \item неподвижный
    \item двухмерное
    \item двухмерный
    \item трехмерное
    \item трехмерный
\end{cutelist}