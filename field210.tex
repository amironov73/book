\chapter{Поле 210: выходные данные}

Соответствует полю RUSMARC: 210

ГОСТ 7.1-2003, п. 5.5

Повторяется для каждого издательства.

ГОД ИЗДАНИЯ (подполе d) — последний элемент выходных данных издания. Им по закону РФ "<Об авторском праве и смежных правах"> от 09.07.1993 N 5351-1 считается год выпуска в обращение экземпляров произведения (издания), т. е. год сдачи тиража или его начальной партии книготорговцу. Указывается арабскими цифрами (без сокращенного или полного слова "<год">. ГОСТ 7.4—95 требует в повторных изданиях проставлять на обороте титульном листе год выпуска предшествующего издания (напр.: 2-е издание вышло в 1975 г.), а в многотомных -- год выпуска первого тома, т. е. начала выпуска всего многотомного издания (напр.: Т. 1 вышел в 1991 г.), однако при контртитуле с общими для всего многотомного издания выходными сведениями целесообразно в выходных данных ставить год выпуска 1-го тома с висячим тире во всех томах, кроме последнего, где указывают год выпуска первого тома и через тире -- последнего.

В периодических (кроме газет) и продолжающихся изданиях год издания указывают при номере издания независимо от года выхода его в свет.

\textbf{Примеры.} \textbf{Один город, два издательства}

\begin{tabular}{| l | l | l |}
	\hline
	\thead{Поле} & \thead{Подполе} & \thead{Значение} \\
	\hline
	\multicolumn{3}{|l|}{text210: Выходные данные} \\
	\hline
	& a: Город1 & Москва \\
	\hline
	& c: Издательство & ИНФРА-М \\
	\hline
	& d: Год издания & 2013 \\
	\hline
	\multicolumn{3}{|l|}{210: Выходные данные -- повторение} \\
	\hline
	& a: Город1 & Москва \\
	\hline
	& c: Издательство & Новое знание \\
	\hline
	& ?: Роль (Города не выводить?) & 1 \\
	\hline
\end{tabular}

\smallskip
Результат
\smallskip

\noindent\fbox{
	\begin{minipage}{\linewidth}
		Москва : ИНФРА-М : Новое знание, 2013
	\end{minipage}
}
\smallskip

\textbf{Два города, два издательства}

\begin{tabular}{| l | l | l |}
	\hline
	\thead{Поле} & \thead{Подполе} & \thead{Значение} \\
	\hline
	\multicolumn{3}{|l|}{210: Выходные данные} \\	
	\hline
	& a: Город1 & Москва \\
	\hline
	& x: Город2 & Санкт-Петербург \\
	\hline
	& c: Издательство & ИНФРА-М \\
	\hline
	& d: Год издания & 2013 \\
	\hline
	\multicolumn{3}{|l|}{210: Выходные данные -- повторение} \\	
	\hline
	& a: Город1 & Москва \\
	\hline
	& x: Город2 & Санкт-Петербург \\
	\hline
	& c: Издательство & Новое знание \\
	\hline
	& ?: Роль (Города не выводить?) & 1 \\
	\hline
\end{tabular}

\smallskip
Результат
\smallskip

\noindent\fbox{
	\begin{minipage}{\linewidth}
		Москва ; Санкт-Петербург : ИНФРА-М : Новое знание, 2013
	\end{minipage}
}
\smallskip

\textbf{Два издательства в двух разных городах}

\begin{tabular}{| l | l | l |}
	\hline
	\thead{Поле} & \thead{Подполе} & \thead{Значение} \\
	\hline
	\multicolumn{3}{|l|}{210: Выходные данные} \\	
	\hline
	& a: Город1 & Москва \\
	\hline
	& c: Издательство & ИНФРА-М \\
	\hline
	& d: Год издания & 2013 \\
	\hline
	\multicolumn{3}{|l|}{210: Выходные данные -- повторение} \\	
	\hline
	& a: Город1 & Санкт-Петербург \\
	\hline
	& c: Издательство & Новое знание \\
	\hline
\end{tabular}

\smallskip
Результат
\smallskip

\noindent\fbox{
	\begin{minipage}{\linewidth}
		Москва : ИНФРА-М ; Санкт-Петербург : Новое знание, 2013
	\end{minipage}
}
