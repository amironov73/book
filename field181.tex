\chapter{Поле 181: коды - вид содержания}

Поле введено в ИРБИС64+ версии 2018.1 в связи с переходом на ГОСТ 7.0.100-2018.

\section{Подполе A: вид содержания}

Значения берутся из справочника \emph{181vs.mnu}.

\begin{cutelist}
    \item \textbf{a} -- электронные данные
    \item \textbf{b} -- изображение
    \item \textbf{c} -- движение
    \item \textbf{d} -- музыка
    \item \textbf{e} -- предмет
    \item \textbf{f} -- программа
    \item \textbf{g} -- звуки
    \item \textbf{h} -- устная речь
    \item \textbf{i} -- текст
    \item \textbf{m} -- разные виды содержания
    \item \textbf{z} -- другой вид содержания
\end{cutelist}

\section{Подполе B: степень применимости}

Значения берутся из справочника \emph{181sp.mnu}.

\begin{cutelist}
    \item \textbf{0} -- не применимо
    \item \textbf{1} -- частично
    \item \textbf{2} -- существенно
    \item \textbf{3} -- преобладает
    \item \textbf{4} -- полностью
\end{cutelist}

\section{Подполе C: спецификация типа}

Значения берутся из справочника \emph{181st.mnu}.

\begin{cutelist}
    \item \textbf{a} -- записанный знаками
    \item \textbf{b} -- исполняемый
    \item \textbf{c} -- картографический
    \item \textbf{x} -- не применяется
\end{cutelist}

\section{Подполе D: спецификация движения}

Используется только с видом содержания "<изображение">.

Значения берутся из справочника \emph{181sd.mnu}.

\begin{cutelist}
    \item \textbf{a} -- движущееся
    \item \textbf{b} -- неподвижное
    \item \textbf{x} -- не применяется
\end{cutelist}

\section{Подполе E: спецификация размерности}

Используется только с видом содержания "<изображения">.

Значения берутся из справочника \emph{181sr.mnu}.

\begin{cutelist}
    \item \textbf{2} -- двухмерное
    \item \textbf{3} -- тр\"eхмерное
    \item \textbf{x} -- не применяется
\end{cutelist}

\section{Подполе F: сенсорная спецификация}

Значения берутся из справочника \emph{181ss.mnu}.

\begin{cutelist}
    \item \textbf{a} -- слуховой
    \item \textbf{b} -- вкусовой
    \item \textbf{с} -- обонятельный
    \item \textbf{d} -- тактильный
    \item \textbf{e} -- визуальный
\end{cutelist}
