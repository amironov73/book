\chapter{Поле 181: коды - вид содержания}

Поле содержит кодированные данные, определяющие вид содержания и характеристику содержания каталогизируемого ресурса, в соответствии с определением элементов Вид содержания и Характеристика содержания Области 0 ISBD и Области вида содержания и средства доступа ГОСТ Р 7.0.100–2018.

Поле введено в ИРБИС64+ версии 2018.1 в связи с переходом на ГОСТ 7.0.100-2018.

Поле факультативное. Повторяется, если ресурс включает несколько видов содержания, либо используется более одной системы кодов.

При необходимости данные о виде содержания могут вводиться в текстовой форме в поле 203. В таком случае поле 203 рекомендуется использовать в дополнение к полю 181 (а не вместо него); индикатор 2 в поле 181 при этом должен иметь значение 0.

Коды в поле 181 могут не соответствовать текстовым данным в поле 203. Порядок генерации вывода определяется в соответствии с практикой учреждения, использующего записи.

Формат допускает использование полей 181 и 182 для автоматической генерации Области 0 ISBD и Области вида содержания и средства доступа ГОСТ Р 7.0.100-2018. Однако, в соответствии с ГОСТ Р 7.0.100-2018, термины характеристики содержания приводят в грамматическом согласовании с терминами, обозначающими вид содержания, поэтому автоматическая генерация Области вида содержания и средства доступа возможна только при условии реализации в информационной системе необходимых механизмов морфологического анализа. В связи с этим рекомендуется для формирования дисплейного представления включать в запись поле 203, а в полях 181 и 182 указывать Инд.2=0.

\section{Подполе A: вид содержания}

Код вида содержания (ISBD / ГОСТ Р 7.0.100–2018).

Две позиции символов определяют категорию основного вида информации, имеющейся в ресурсе, в соответствии с определениями ISBD / ГОСТ Р 7.0.100–2018 для этого элемента, а также степень применимости указанного вида содержания для каталогизируемого ресурса. Все данные, записываемые в подполе A, идентифицируются позицией символа в подполе. Позиции символов нумеруются от 0 до 1.

Подполе факультативное. Не повторяется.

Значения берутся из справочника \emph{181vs.mnu}.

\begin{cutelist}
    \item \textbf{a} -- электронные данные
    \item \textbf{b} -- изображение
    \item \textbf{c} -- движение
    \item \textbf{d} -- музыка
    \item \textbf{e} -- предмет
    \item \textbf{f} -- программа
    \item \textbf{g} -- звуки
    \item \textbf{h} -- устная речь
    \item \textbf{i} -- текст
    \item \textbf{m} -- разные виды содержания
    \item \textbf{z} -- другой вид содержания
\end{cutelist}

Элементы данных фиксированной длины подполя A:

{\noindent\begin{tabular}{|p{5cm}|p{25mm}|p{25mm}|}
    \hline 
    \thead{Наименование элемента данных} & \thead{Количество символов}  & \thead{Позиции символов} \\ 
    \hline 
    Вид содержания & 1  &  0 \\
    \hline 
    Степень применимости & 1 & 1 \\
   \hline
\end{tabular}}

\section{Подполе B: степень применимости}

Код характеристики содержания (ISBD / ГОСТ Р 7.0.100–2018).

Шесть позиций символов определяют характеристики содержания, которые применимы к каталогизируемому ресурсу, в соответствии с определениями ISBD / ГОСТ Р 7.0.100–2018 для этого элемента. Все данные, записываемые в подполе B, идентифицируются позицией символа в подполе. Позиции символов нумеруются от 0 до 5.

Подполе факультативное. Повторяется.

{\noindent\begin{tabular}{|p{5cm}|p{25mm}|p{25mm}|}
        \hline 
        \thead{Наименование элемента данных} & \thead{Количество символов}  & \thead{Позиции символов} \\ 
        \hline 
        Спецификация природы информации & 1  &  0 \\
        \hline 
        Спецификация наличия или отсутствия движения & 1 & 1 \\
        \hline 
       Спецификация размерности & 1 & 2 \\
        \hline 
        Спецификация способа сенсорного восприятия & 1 & 3-5 \\
        \hline
\end{tabular}}

Значения берутся из справочника \emph{181sp.mnu}.

\begin{cutelist}
    \item \textbf{0} -- не применимо
    \item \textbf{1} -- частично
    \item \textbf{2} -- существенно
    \item \textbf{3} -- преобладает
    \item \textbf{4} -- полностью
\end{cutelist}

\section{Подполе C: спецификация типа}

Значения берутся из справочника \emph{181st.mnu}.

\begin{cutelist}
    \item \textbf{a} -- записанный знаками
    \item \textbf{b} -- исполняемый
    \item \textbf{c} -- картографический
    \item \textbf{x} -- не применяется
\end{cutelist}

\section{Подполе D: спецификация движения}

Используется только с видом содержания "<изображение">.

Значения берутся из справочника \emph{181sd.mnu}.

\begin{cutelist}
    \item \textbf{a} -- движущееся
    \item \textbf{b} -- неподвижное
    \item \textbf{x} -- не применяется
\end{cutelist}

\section{Подполе E: спецификация размерности}

Используется только с видом содержания "<изображения">.

Значения берутся из справочника \emph{181sr.mnu}.

\begin{cutelist}
    \item \textbf{2} -- двухмерное
    \item \textbf{3} -- тр\"eхмерное
    \item \textbf{x} -- не применяется
\end{cutelist}

\section{Подполе F: сенсорная спецификация}

Значения берутся из справочника \emph{181ss.mnu}.

\begin{cutelist}
    \item \textbf{a} -- слуховой
    \item \textbf{b} -- вкусовой
    \item \textbf{с} -- обонятельный
    \item \textbf{d} -- тактильный
    \item \textbf{e} -- визуальный
\end{cutelist}
