\chapter{ГОСТ 7.0.100-2018}

% TODO начало

Причины разработки стандарта:

\begin{cutelist}
    \item Изменения с момента введения ГОСТ 7.1-2003.
    \item Новые требования к информации в машиночитаемой форме.
    \item Объединение в едином стандарте документов текстовых и электронных.
    \item Уч\"eт норм Международного стандартного библиографического описания.
\end{cutelist}

Базовые новации стандарта:

\begin{cutelist}
    \item Впервые ГОСТ стал национальным стандартом.
    \item Изменилась структура библиографического описания по количеству, названию и составу областей и элементов.
    \item Введена 9-я область "<Область вида содержания и средства доступа">.
    \item Элемент "<Общее обозначение материала"> исключ\"eн.
    \item Введ\"eн обязательный элемент "<Примечание для электронных и депонированных ресурсов">.
\end{cutelist}

Состав библиографического описания

\noindent\begin{tabular}{|p{5cm}|p{5cm}|}
    \hline 
    \thead{ГОСТ Р 7.0.100} & \thead{ГОСТ 7.1-2003} \\ 
    \hline 
    область заглавия и сведений об ответственности (О) &  область заглавния и сведений об ответственности (О) \\ 
    \hline 
    область издания (О) & область издания (О) \\ 
    \hline 
    область публикации, производства, распространения и т. д. (О) & область выходных данных (О) \\ 
    \hline 
    область физической характеристики (О) & область физической характеристики (О) \\ 
    \hline 
    область серии и многочастного монографического ресурса (О) & область серии (О) \\ 
    \hline 
    область примечания (О для электронных и депонированных ресурсов) & область примечания \\ 
    \hline 
    область идентификатора ресурса и условий доступности (О) & область стандартного номера (или его альтернативы) и условий доступности (О) \\ 
    \hline 
    область вида содержания и средства доступа &  \\ 
    \hline 
\end{tabular} 

Статус элементов библиографического описания: обязательные, условно-обязательные и факультативные

\begin{cutelist}
    \item \textbf{краткое} библиографическое описание -- только обязательные элементы;
    \item \textbf{расширенное} библиографическое описание -- обязательные и условно-обязательные элементы;
    \item \textbf{полное} библиографическое описание -- обязательные, условно-обязательные и факультативные элементы.
\end{cutelist}

\textbf{Схема кратного описания.}

\textbf{Основное заглавие} / Первые сведения об ответственности. -- Сведения об издании (и Дополнительные сведения об издании). -- Сведения о масштабе (или Сведения о форме изложения нотного текста для нотных ресурсов, Сведения о нумерации для сериальных изданий)ю -- Первое место публикации : Имя издателя, производителя и/или распространителя, Дата публикации, производства и/или распространения. -- Специфическое обозначение материала и объ\"eм. -- (Основное заглавие серии/подсерии или многочастного монорафического ресурса, Международный стандартный номер серии/подсерии или многочастного монографического ресурса ; Номер выпуска серии/подсерии или многочастного монографического ресурса). -- Примечания (только для электронных и депонированных ресурсов). -- Международный стандартный номер.

Некторые примечания являются обязательными

\begin{cutelist}
    \item для электронных локальных ресурсов обязательным является примечание об источнике основного заглавия\\
    \textsl{. -- Загл. с титул. экрана}
    \item для электронных ресурсов сетевого распространения обязательным является примечание об электронном адресе ресурса в сети Интернет и дате обращения\\
    \textsl{. -- URL: \underline{http://government.ru} (дата обращения: 19.02.2018)}
    \item для депонированных ресурсов обязательными являются сведения о депонировании\\
    \textsl{. -- Деп. в ВИНИТИ 18.05.2017, № 14432}
\end{cutelist}

Полный набор обязательных, условно-обязательных и факультативных элементов приводят в описаниях для \textbf{государственных} библиографических указателей, библиотечных каталогов, банков и баз данных \textbf{национальных библиотек}, центров государственной библиографии (п. 4.4.4 ГОСТ).

Все данные в библиографическом описании могут быть представлены в полной форме.

\textbf{Предписанные источники информации.} Для текстовых изданий предписанным источником, как и в прежнем ГОСТ, является \textbf{титульный лист} (титульная страница и оборот титульного листа). Квадратные скобки для сведений, взятых с оборота титульного листа, не ставятся.

\textbf{Область заглавия и сведений об ответственности}. Основные изменения:

\begin{cutelist}
    \item Элемент "<Общее обозначение материала"> удалён, его заменила с переносом в конец "<Область вида содержания и средства доступа">;
    \item Сокращения слов минимизированы;
    \item Сведения, относящиеся к заглавию, стали условно-обязательным элементом;
    \item Другое заглавие приводят в описании в качестве сведений, относящихся к заглавию\\
    \textsl{Памятник Петру I : Медный всадник}
    \item Количество лиц и организаций в сведениях об ответственности увеличилось, но они не являются обязательными. Сведения об ответственности являются обязательными только для первых (до знака ;);
    \item Добавлены элементы, которые ранее были в области специфических сведений (для нормативных изданий).
\end{cutelist}

\textbf{Сведения об ответственности}

\begin{cutelist}
    \item Отменено "<правило тр\"eх">: в описании могут быть приведены сведения обо всех лицах и/или организациях, указанных в источнике информации;
    \item \textbf{Допускается сокращать количество приводимых сведений.} Можно указать имена одного, двухх, тр\"eх или четыр\"eх авторов, а при наличии информации о пяти и более авторах -- имена первых тр\"eъ и в квадратных скобках сокращение "<{[}и др.{]}">\\
    \textsl{/ А. В. Мельников, В. А. Степанов, А. С. Вах {[}и др.{]}}
    \item При наличии информации о тр\"eх и более организациях приводят наименование первой и в квадратных скобках сокращение "<{[}и др.{]}">\\
    \textsl{/ Благотворительный фонд Потанина {[}и др.{]}}
\end{cutelist}

\textbf{Область издания}

Область содержит информацию об изменениях и особенностях данного издания по отношению к предыдущему изданию того же произведения. \textbf{Не изменилась.}

\textsl{. -- Изд. 6-е, испр. и доп.}

\textsl{. -- {[}Новое изд.{]}}

\textbf{Специфическая область материала или вида ресурса}

Исключены электронные ресурсы и отдельные виды нормативных и технических документов -- стандарты, патенты и т. п., их специфические элементы размещаются в других областях и элементах, в основном в сведениях, относящихся к заглавию

\textsl{: введен впервые : дата введения 2018-05-01}

Специфическую область материала или вида ресурса используют только при описании картографических, нотных, сериальных ресурсов

\textsl{. -- Партитура и клавир}

\textbf{Область публикации, производства, распространения и т. д.}

Новое название области выходных данных. Практически без изменений

\textsl{. -- Санкт-Петербург : БАН, 2017}

Сведения об издателе могут быть опущены для газет, журналов, сайтов

\textsl{{[}б. и.{]}} -- не приводят

\textbf{Область физической характеристики}

Без особых изменений

\textbf{Область серии и многочастного монографического ресурса}

Международный номер стал обязательным элементом.

Международный номер (ISSN, ISBN, ISMN и др.) приводят, если он указан в ресурсе или установлен по источникам вне ресурса.

\textsl{. -- (Вопросы атомной науки, ISSN 0557-6733 ; вып. 70)}

\textsl{. -- (Домосед ; т. 4, ч. 2)}

\textsl{. -- Партитура и клавир}

\textbf{Область примечания (для электронных ресурсов и патентов)}

Обязательна для электронных ресурсов и патентов.

Для сетевых электронных ресурсов обязательно примечание об электронном адресе и дате обращения, условно-обязательно -- о режиме доступа

\textsl{. -- URL: \underline{http://www.rba.ru} (дата обращения: 14.04.2018)}

Для электронных локальных ресурсов обязательны системные требования и сведения об источнике основного заглавия (п. 5.8.6.3 ГОСТ)

\textsl{. -- Загл. с этикетки видеодиска}

Дату публикации в электронных журналах приводят вместо даты обращения (она обязательна)

\textsl{. -- URL: \underline{http://www.nillc.ru/journal/} -- Дата публикации: 21.04.2017}

\textbf{Область идентификатора ресурса и условий доступности}

Уточнено название области.

Введ\"eн обобщающий термин -- идентификатор. Это обязательный элемент.

\textsl{. -- ISBN 978-5-84213-911-0}

\textsl{. -- ISSN 1563-0102}

\textsl{. -- DOI 10.1596/978-0-8213-6475-8}

\textsl{. -- № гос. регистрации 0321701986}

\textbf{Область вида содержание и средства доступа}

Условно обязательная новая область № 9 содержит сведения о природе информации ресурса и средстве, обеспечивающем доступ к нему.

В ГОСТ перечислены термины для обозначения вида содержания.

\textsl{. -- Изображение. Текст}

\textsl{. -- Текст : непосредственный}

\textsl{. -- Текст : аудио}

\textbf{Составные части ресурсов}

Сведения приводят по прежней схеме:

\textsl{. -- Сведения о составной части ресурса // Сведения об идентифицирующем ресурсе. -- Сведения о местоположении составной части в ресурсе. -- Примечания.}

\textbf{Новация:} в сведения об идентифицирующем ресурсе (п. 7.3.7) может быть включено и имя издателя.

\textsl{. -- Москва : РКП, 2017.}

\textbf{Примеры библиографических записей для списка литературы}

\textbf{Колтухова, И. М.} Классика и современная литература: почитаем и подумаем вместе \textit{: учебно-методическое пособие} / И. М. Колтухова ; \textit{Министерство образования и науки Российской Федерации, Крымский федеральный университет им. В. И. Вернадского, Таврическая академия, Факультет славянской филологии и журналистики, Кафедра методики преподавания филологических дисциплин.} -- Симферополь : Ариал, 2017. -- 151 с. -- \textit{Библиогр.: с. 149-151.} -- ISBN 978-5-906962-43-0. -- \textit{Текст : непосредственный}.

Странные истории : \textit{{[}для лиц старше 16 лет{]}} / \textit{перевод с английского И. Гуровой {[}и др.{]}.} -- Москва ; Тверь : Мартин, 2017. -381, {[}2{]} с. -- (Избранная классика. Pocket-book). -- \textit{Содерж.: Странная история доктора Джекила и мистера Хайда / Р. Стивенсон. Портрет Дориана Грея / О. Уайлд. Остров доктора Моро / Г. Уэллс.} -- ISBN 978-5-8475-1050-9. -- \textit{Текст : непосредственный.}

\textbf{Российская Федерация. Законы. Об общих принципах организации местного самоуправления в Российской Федерации :} \textit{Федеральный закон № 131-Ф№ : {[}Принят Государственной думой 16 сентября 2003 года : одобрен Советом Федерации 24 сентября 2003 года{].}} -- Москва ? Проспект ; Санкт-Петербург : Кодекс, 2017. -- 158 с. -- ISBN 978-5-392-26365-3. -- Текст : непосредственный.

\textbf{Голсуорси, Д.} Сага о Форсайтах : \textit{{[}в 2 томах{]}} / Джон Голсуорси ; \textit{перевод с английского М. Лорие {[}и др.{]}.} -- Москва : Время, 2017. -- 2 т. -- (Сквозь время). -- ISBN 978-5-00112-035-3. -- \textit{Текст : непосредственный}.

\textbf{Жукова, Н. С.} Инженерные системы и сооружения. Учебное пособие. В 3 частях. Часть 1. Отопление и вентиляция / Н. С. Жукова, В. Н. Азаров ; \textit{Министерство образования и науки Российской Федерации, Волгоградский государственный технический университет.} -- Волгоград : ВолГТУ, 2017. -- 89, {[}3{]} с. -- \textit{Библиогр.: с. 92}. -- ISBN 978-5-9948-2526-6. -- \textit{Текст : непосредственный.}

\textbf{Ратнер, Л. Н.} Дорогой великой скорби : памяти новомучеников : \textit{{[}комплект репродукций графических работ{]}} / Лилия Ратнер ; \textit{автор статьи И. Языкова}. -- Москва : МХК "Осанна", 2017. -- 1 папка (17, {[}1{]} отд. л.) : \textit{{\color{red}цв. ил. ; 30x22 см}}. -- ISBN 978-5-901293-09-6. -- Изображение : непосредственное.

Литературная Москва 100 лет назад : календарь : 2017 / авторы-составители: О. Лекманов, Ф. Лекманов ; \textit{художественное оформление: А. Рыбаков}. -- Москва : Б.С.Г.-Пресс, 2016. -- {[25]} с. : \textit{{\color{red} цв. ил. ; 59x43 см}}. -- ISBN 978-5-93381-371-2. -- \textit{Изображение : непосредственное}.

\vspace{2cm}

В процессе дискуссии собравшиеся обсудили также:

* нерешённые новым стандартом проблемы в описании опубликованных и неопубликованных документов сетевого распространения,

* нормы стандарта, регламентирующие размещение знака информационной продукции в условно-обязательном элементе "<Сведения, относящиеся к заглавию"> (в п. 5.2.5.3 ГОСТа данный знак фигурирует только в примерах описания),

* вопросы, касающиеся наличия полей в программном обеспечении электронных каталогов, позволяющих разместить сведения из новой области описания "<Область вида содержания и средства доступа">.

\vspace{2cm}

% TODO кончало

Согласно п. 4. 3 в состав библиографического описания входят следующие области в привед\"eнной ниже последовательности:

\begin{cutelist}
    \item область заглавия и сведений об ответственности;
    \item область издания;
    \item специфическая область материала или вида ресурса;
    \item область публикации, производства, распространения и т. д.;
    \item область физической характеристики;
    \item область серии и многочастного монографического ресурса;
    \item область примечания;
    \item область идентификатора ресурса и условий доступности;
    \item область вида содержания и средства доступа.
\end{cutelist}

Пункт 4. 4: области описания состоят из элементов, которые делятся на обязательные, условно обязательные и факультативные. В зависимости от набора элементов различают:

\begin{cutelist}
    \item краткое библиографическое описание (содержит только обязательные элементы);
    \item расширенное библиографическое описание (содержит обязательные и условно-обязательные элементы);
    \item полное библиографическое описание (содержит обязательные, ус\-лов\-но-обязательные и факультативные элементы).
\end{cutelist}

