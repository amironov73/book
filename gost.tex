\chapter{ГОСТ 7.0.100-2018}

Согласно п. 4. 3 в состав библиографического описания входят следующие области в привед\"eнной ниже последовательности:

\begin{itemize}
    \item область заглавия и сведений об ответственности;
    \item область издания;
    \item специфическая область материала или вида ресурса;
    \item область публикации, производства, распространения и т. д.;
    \item область физической характеристики;
    \item область серии и многочастного монографического ресурса;
    \item область примечания;
    \item область идентификатора ресурса и условий доступности;
    \item область вида содержания и средства доступа.
\end{itemize}

Пункт 4. 4: области описания состоят из элементов, которые делятся на обязательные, условно обязательные и факультативные. В зависимости от набора элементов различают:

\begin{itemize}
    \item краткое библиографическое описание (содержит только обязательные элементы);
    \item расширенное библиографическое описание (содержит обязательные и условно-обязательные элементы);
    \item полное библиографическое описание (содержит обязательные, условно-обязательные и факультативные элементы).
\end{itemize}

