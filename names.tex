\chapter{Системы личных имён у народов мира}

\section{Испанские (иберийские) фамилии}

В испано- и португалоязычных регионах планеты существуют сходные правила построения им\"eн, эти правила берут начало из ономастических традиций Испании и Португалии, известных под общим названием \emph{иберийское имя}.

В большинстве случаев иберийское имя состоит из составного личного имени и хотя бы двух фамилий. В официальных документах (паспорт, титул на собственность) используется полное имя, но в повседневном обороте почти всегда используются часть личного имени и лишь одна из фамилий.

\emph{Гарсиа Лорка, Федерико} -- здесь \emph{Гарсиа} -- фамилия отца, а \emph{Лорка} -- фамилия матери.

\emph{Гарсиа Маркес, Габриэль Хосе де ла Конкордиа} -- здесь \emph{Гарсиа} -- фамилия отца, а \emph{Маркес} -- фамилия матери. Иностранцы, в том числе англоязычные и русско-язычные, очень часто считают фамилией только последнюю часть имени. Это приводит к неверным сокращениям, если по какой-то причине используются обе фамилии. В частности, писатель \emph{Габриэль Гарсиа Маркес} подписывал книги обеими фамилиями \emph{Гарсиа Маркес}, по-видимому, просто потому что фамилия \emph{Гарсиа} очень распространена в испаноязычном мире. Но в статьях и обзорах на других языках его фамилию сокращают до \emph{Маркес}, что неверно, хотя уже и вошло в русскоязычной литературе в традицию.

\emph{Ортега-и-Гассет, Хосе} -- здесь \emph{Ортега} -- фамилия отца, а \emph{Гассет} -- фамилия матери. В русском языке служебный элемент "<\emph{и}"> в испанских фамилиях выделяется дефисами: \emph{Хосе Ортега-и-Гассет}, \emph{Риего-и-Нуньес}. В оригинальном испанском написании этих дефисов нет.

\emph{Варгас Льоса, Хорхе Марио Педро} -- здесь \emph{Варгас} -- фамилия отца, а \emph{Льоса} -- фамилия матери.

Испанцы ставят фамилию отца перед фамилией матери, португальцы и бразильцы — наоборот, однако порядок может измениться. Также между именами могут появиться короткие слова, как то: \emph{de} или \emph{e} между словами: \emph{Carre\~no de Qui\~nones}, \emph{Tavares e Silva}.

\emph{Сантуш, Жозе Эдуарду душ} (\emph{Jos\'e Eduardo dos Santos}) -- второй президент Анголы.

Части сложного имени на португальском пишутся раздельно, как и части сложной фамилии. В русской транскрипции все части имени и (отдельно) фамилии соединяются дефисами:

\emph{Jose Manuel Dur\~ao Barroso} записывается как \emph{Жозе-Мануэл Дуран-Баррозу}.

У бразильцев также бывает до четырех фамилий, наследуемых от предков, например \emph{Jos\'e Eduardo Santos Tavares Melo Silva}.

Служебные слова в фамилиях:

{\noindent\small
    \begin{tabularx}{\linewidth}{|X|X|X|}
        \hline 
        \thead{Служебное слово с транскрипцией} & \thead{Пример} & \thead{Русская транскрипция} \\ 
        \hline 
        da / да & da Costa & да-Кошта \\ 
        \hline 
        de / ди & Lopes de Castanheda & Лопиж-ди-Каштаньеда \\ 
        \hline 
        do / ду & Ferreira do Amaral & Феррейра-ду-Амарал \\ 
        \hline 
        \raggedright dos /душ (перед глухими согласными) & dos Santos & душ-Сантуш \\ 
        \hline 
        \raggedright дуж (перед звонкими согласными) & dos Reis & дуж-Рейш \\ 
        \hline 
        e / и & de Andrada e Silva & ди-Андрада-и-Силва \\ 
        \hline 
    \end{tabularx} 
}

\section{Римские фамилии}

В классическое время полное римское мужское имя обычно состояло из тр\"eх компонентов: личного имени, или \emph{преномена} (\emph{praenomen}), родового имени, или \emph{номена} (\emph{nomen}), и индивидуального прозвища или наименования ветви рода, \emph{когномена} (\emph{cognomen}).

Личное имя было подобно современному мужскому имени. Римляне употребляли небольшое число личных имен (18 им\"eн из общего количества 72): \emph{Аппий, Авл, Децим, Гай, Гней, Кезон, Луций, Мамерк, Маний, Марк, Нумерий, Публий, Квинт, Сервий, Секст, Спурий, Тит, Тиберий}.

Родовое имя было названием рода и соответствовало, приблизительно, современной фамилии. Указывалось в форме прилагательного мужского рода и оканчивалось в классическую эпоху на \emph{-ius}: \emph{Tullius} -- \emph{Туллий} (из рода \emph{Туллиев}), \emph{Julius} -- \emph{Юлий} (из рода \emph{Юлиев}); в республиканское время встречаются также окончания \emph{-is}, \emph{-i}. Родовые имена неримского происхождения имели отличные от названных окончания.

\emph{Публий Корнелий Тацит} -- древнеримский историк (\emph{Тацит, Корнелий}).

\section{Греческие фамилии}

Греческое имя, в соответствии с антропонимической моделью греков, состоит из тр\"eх частей, идущих в следующем порядке: индивидуальное имя, имя отца в родительном падеже и фамилия.

\emph{Ангелос Терзакис} -- греческий романист и драматург (\emph{Терзакис, Ангелос}).

\emph{Поликлет Старший} -- теоретик искусства (\emph{Поликлет Старший}).

\emph{Филон из Мегары} (\emph{Филон Диалектик}) -- древнегреческий философ (\emph{Филон из Мегары}).

\section{Итальянские фамилии}

\emph{Данте Алигьери} -- итальянский поэт (\emph{Данте Алигьери}).

\emph{Джованни делла Каза} -- итальянский писатель (\emph{Каза, Джованни делла}).

\emph{Кола ди Риенцо} -- итальянский политический деятель (\emph{Риенцо, Кола ди}).

\emph{Фердинандо Петруччелли делла Гаттина} -- итальянский журналист (\emph{Петруччелли делла Гаттина, Фердинандо}).

\emph{Джанфранко Миро Гори} -- итальянский поэт и эссеист (\emph{Гори, Джанфранко Миро}).

\section{Французские имена}

Французское законодательство позволяет человеку иметь несколько личных имён. Только одно из них (как правило, первое) используется в повседневной практике, остальные -- только в официальных документах, таких как свидетельства о рождении, смерти и браке. Не путать с составными именами католической традиции: \emph{Жан-Клод, Жан-Жак}. Такие конструкции являются одним (единым и неделимым) именем. \emph{Жан-Клода} ни при каких обстоятельствах не назовут ни \emph{Жаном}, ни \emph{Клодом}.

\emph{Антуан Мари Жан-Батист Роже де Сент-Экзюпери} -- французский писатель и журналист (\emph{Сент-Экзюпери, Антуан де}).

\emph{Поль Шарль Филипп Эрик Дел\"eтр} (\emph{Поль д’Ивуа}) – французский писатель (\emph{Д’Ивуа, Поль}).

\emph{Шарль Ле Гоффик} -- французский поэт (\emph{Ле Гоффик, Шарль}).

\emph{Пьер Дри\"e ла Рошель} -- французский писатель (\emph{Дри\"e ла Рошель, Пьер}).

\section{Английские имена}

Традиционно в англоговорящих странах реб\"eнок при рождении получает два имени: личное имя (англ. \emph{personal name}, \emph{first name}) и среднее имя (англ. \emph{middle name}). Наиболее важным, существенным представляется именно первое, личное имя.

\emph{Артур Игнейшус Конан Дойл} -- английский писатель (\emph{Дойл, Артур Конан}).

\emph{Герберт Джордж Уэллс} -- английский писатель и публицист (\emph{Уэллс, Герберт Джордж}).

\emph{Джон Рональд Руэл Толкин} -- английский писатель и поэт (\emph{Толкин, Джон Рональд Руэл}).

\emph{Уолтер Джон Де Ла Мар} -- английский поэт (\emph{Де Ла Мар, Уолтер Джон}).

\section{Ирландские фамилии}

Ирландские фамилии и имена отражают разнообразие традиций, языков, которые были интегрированы в современную ирландскую культуру. Ирландские личные имена обычно берут своё начало в древних кельтских именах, кельтской христианской традиции, и аглицированных формах гэльских им\"eн.

В фамилиях используются префиксы \emph{Мак} и \emph{О'}: Маккартни (англ. \emph{McCartney}), \emph{О'Салливан} (англ. \emph{O'Sullivan}), \emph{О'Брайен} (англ. \emph{O'Brien}).

\section{Исландские имена и отчества}

Исландское имя состоит из совокупности имени, отчества (патронима) и в очень редких случаях -- фамилии. Большинство исландцев имеют только имя и отчество (подобная практика существовала прежде и в скандинавских странах). В обращении и при упоминании человека употребляется только имя, независимо от того, обращается ли говорящий к данному лицу на "<вы"> или на "<ты">. Отчество представляет собой сложную конструкцию, состоящую из имени отца в родительном падеже и слов "<сын"> (исл. \emph{son}, русский аналог "<-вич">) -- для мужчин -- или "<дочь"> (исл. \emph{d\'ottir}, русский аналог "<-вна">) -- для женщин. Например, \emph{Лейф Эрикссон} (исл. \emph{Leifur Eiriksson}), \emph{Снорри Стурлусон} (исл. \emph{Snorri Sturluson}), \emph{Ингибь\"eрг Стефаунсдоуттир} (исл. \emph{Ingibj\"org Stef\'ansd\'ottir}), \emph{Бьорк Гвюдмюндсдоуттир} (\emph{Bj\"ork Gudmundsd\'ottir}). Иногда вместо патронима используется матроним -- форма, производная от имени матери. Например, \emph{Хейдар Хельгюсон} (сын \emph{Хельги}), \emph{Вигдис Хельгюдоуттир} (дочь \emph{Хельги}).

\section{Голландские фамилии}

Именование людей в Нидерландах соответствует способам, принятым в большинстве европейских стран: человек имеет одно или несколько им\"eн, и фамилию. Многие нидерландские фамилии имеют префикс: \emph{van} ("<из">), \emph{de} (артикль), \emph{het} (артикль), \emph{'t} (артикль), \emph{der}, \emph{van de}, \emph{van der}, \emph{van den}, \emph{te}, \emph{ter}, \emph{in het}. В Нидерландах при использовании с именем или инициалом эти префиксы не выделяются заглавными буквами: \emph{Piet de Wolff}. Заглавные буквы, однако, полагается использовать при отсутствии имени -- после обращения или академической степени: \emph{de heer Van Kampen}, \emph{dr. Van Wijk}.

В Бельгии система использования заглавных букв отличается. В большинстве случаев в Бельгии во фламандских фамилиях с заглавной буквы пишутся все слова, хотя есть различия в использовании префикса \emph{van de}: в одних фамилиях он пишется \emph{Van de}, в других -- \emph{Van De}. Кроме того, префикс может сливаться с самой фамилией (\emph{Vandecasteele}), писаться отдельно от не\"e (\emph{Van De Casteele}) или может иметь место промежуточный случай (\emph{Vande Casteele}). Этими вариантами фамилии отличаются друг от друга, и не все фамилии существуют в нескольких вариантах.

\emph{Р\"eйсдал, Якоб Исаакс ван} -- художник-пейзажист (\emph{Р\"eйсдал, Якоб ван}) (\emph{Рейсдаль, Якоб ван}).

\emph{Антонис ван Дейк} -- фламандский живописец (\emph{Ван Дейк, Антонис}).

\emph{Корнелиус Вандербильт} -- американский предприниматель (\emph{Вандербильт, Корнелиус}). Его предки голландского происхождения имели фамилию \emph{van der Bilt}.


\section{Венгерские фамилии}

Венгерские имена выделяются на фоне всех остальных именных моделей Европы. Их особенностью является восточный порядок следования имени и фамилии (характерный для Китая, Кореи и Японии), при котором фамилия предшествует имени. 

\emph{Бартиш, Атилла} -- венгерский писатель.

\emph{Мештерхази, Лайош} -- венгерский писатель.

\section{Литовские фамилии}

Окончания женских и мужских фамилий в Литве, как и в России, различаются: у мужчины по фамилии \emph{Катилюс} есть сестра по фамилии \emph{Катилюте}, у \emph{Даукантаса} -- \emph{Даукантайте}.

При этом у замужних женщин суффикс будет другим: например, девушка по фамилии \emph{Варнате} после замужества станет \emph{Варнене}.

Также с 2003 года женщины могут вообще не использовать суффиксы-индикаторы семейного положения.

Что касается им\"eн, то они в Литве бывают двойные и одинарные.

\section{Арабские имена}

Арабская система им\"eн считается одной из наиболее сложных среди современных традиций именования: большинство арабов не имеют простого имени, состоящего из личного имени и фамилии, но обладают длинной цепочкой имён.

Арабское \emph{Абу Карим Мухаммед аль-Джамил ибн Нидал ибн Абдулазиз ал-Филистини} переводится как «Отец Карима, Мухаммед (личное имя), Краси-вый, Сын Нидала, Сын Абдулазиза, Палестинец», при этом \emph{Карим} -- первый сын \emph{Мухаммеда}.

\emph{Омар Хайям} -- \emph{Гийясаддин Абу-ль-Фатх Омар ибн Ибрахим аль-Хайям Нишапури}. \emph{Гийяс ад-Дин} -- \emph{хитаб}, "<помощь религии">, \emph{Абу-ль-Фатх} -- \emph{кунья}, "<отец Фатха">, \emph{Омар} — \emph{исм} (личное имя), \emph{ибн Ибрахим} -- \emph{насаб}, "<сын Ибрахима">, \emph{Хайям} -- \emph{тахаллус}, "<палаточный мастер"> (предположительно, указание на ремесло отца; от слова \emph{хайма} -- палатка, от этого же слова предположительно происходит старорусское "<хамовник"> -- текстильщик), \emph{Нишапури} -- нисба, "<из Нишапура">. (Следует вводить: \emph{Омар Хайям}).

\emph{Саддам Хусейн Абд аль-Маджид ат-Тикрити} -- у \emph{Саддама} (арабское имя \emph{Саддам} означает "<противостоящий">) не было фамилии в европейском понимании. \emph{Хусейн} -- это имя его отца (\emph{насаб}), аналогичное русскому отчеству; \emph{Абд аль-Маджид} -- имя его деда, а \emph{ат-Тикрити} -- \emph{нисба}, указывающая на город Тикрит, откуда \emph{Саддам} родом. Но мы пишем \emph{Хусейн, Саддам}.

\begin{figure}
    \centering
    \includegraphics[width=0.7\linewidth]{img/haiyam}
    \caption{Расшифровка полного имени \emph{Омара Хайяма}}
    \label{fig:haiyam}
\end{figure}

\section{Китайские имена}

Именование человека в китайской, а также связанных с ней культурах отличается от системы имён, принятой на Западе. Наиболее заметным признаком этого различия является то, что в китайском полном имени сначала пишется фамилия, и только после не\"e -- личное имя.

В русском языке между китайскими фамилией и именем обычно ставится пробел: \emph{Фамилия Имя}, при этом имя пишется слитно. В старых источниках китайские имена записывали через дефис (\emph{Фэн Юй-сян}), однако позднее стало принятым слитное написание (правильно -- \emph{Фэн Юйсян}).

\emph{Детское имя}. Например, \emph{Ли Чжэньфань} (\emph{Брюс Ли}) имел детское имя \emph{Ли Сяолун} (\emph{Ли Маленький Дракон}), ставшее впоследствии его прозвищем.

\emph{Второе имя} -- это имя, даваемое при достижении совершеннолетия, которое используются в течение жизни. После 20 лет второе имя да\"eтся как символ взросления и уважения. Первоначально такие имена использовались после мужских им\"eн, человек мог получить второе имя от родителей, от первого учителя в первый день посещения семейной школы, либо мог сам избрать для себя второе имя. Традиция использования вторых им\"eн постепенно начала исчезать со врем\"eн Движения 4 Мая (1919). 

\emph{Прозвание}, \emph{хао} -- это альтернативное второе имя, которое обычно используется в качестве псевдонима. Чаще всего оно состоит из тр\"eх или четыр\"eх иероглифов и, возможно, изначально стало популярным, потому что у многих людей часто были одинаковые вторые имена. Люди чаще всего выбирали \emph{хао} сами себе и могли иметь больше одного прозвища. \emph{Хао} никак не было связано с именем, данным человеку при рождении и его вторым именем; скорее, прозвище было чем-то личным, иногда эксцентричным.

В китайском имени \begin{CJK*}{UTF8}{gbsn} 毛泽东 \end{CJK*}  (\emph{Мао Цзэдун}) фамилией является \emph{Мао}, то есть первое при чтении слева направо. Личным именем является \emph{Дун}. Средний же слог \emph{Цзэ} является именем поколения, и един для всех его братьев и сестер: \emph{Мао Цзэминь}, \emph{Мао Цзэтань}, \emph{Мао Цзэхун}.

Хотя на данный момент не у всех есть имя поколения, особенно в континентальном Китае, те, у кого оно есть, будут считать должным использование его вместе с личным именем.

Китайская система им\"eн является основой для всех традиционных способов именования людей в Восточной Азии.

Китайцы, контактируя с представителями западной цивилизации, могут добавлять дополнительное личное имя, которое будет привычней для их новых друзей. К примеру, \emph{Яо Мин} (фамилия -- \emph{Яо}, личное имя -- \emph{Мин}) может назваться иностранцам как \emph{Фред Яо Мин} или \emph{Фред Мин Яо}.


\section{Тибетские имена}

Тибетская система им\"eн принципиально отличается от китайской и ориентирована в большей степени на Индию. В Тибете нет фамилий. Многие имена являются калькой с санскрита, но есть и традиционные (напр.: \emph{Дава} (тиб. \emph{луна, понедельник}), \emph{Ньима} (тиб. \emph{солнце, воскресенье})).

\emph{Джамьянг Шэпа} -- тибетский уч\"eный.

\section{Корейские имена}

Корейское имя состоит из фамилии и следующего после него личного имени.

В большинстве случаев фамилия состоит из одного слога, а имя из двух слогов. Как имя, так и фамилия часто записываются с помощью \emph{ханча} -- китайских иероглифов, отражающих корейское произношение. Ханча более не используются в Северной Корее, а их использование для имён в Южной Корее сокращено до 5038 иероглифов. При использовании европейских языков некоторые корейцы сохраняют традиционный порядок написания, а другие меняют его согласно западной схеме.

В Корее используется всего около 250 фамилий. Самыми распространёнными из них являются \emph{Ким}, \emph{Ли} и \emph{Пак}. Однако большинство однофамильцев не являются близкими родственниками. Происхождение корейских фамилий тесно связано с корейской историей и географией.

\emph{Ким Дон Ин} -- корейский писатель.

\emph{Ким Юджон} -- корейский писатель.

\emph{Пак Кю Су} -- корейский государственный деятель и писатель.

% TODO Японские имена

% TODO Индийские имена

\section{VIAF}

Для выяснения форм написания имени и фамилии рекомендуется использовать сайт \underline{http://viaf.org} (англ. \emph{Virtual International Authority File} — виртуальный международный авторитетный файл) -- виртуальный каталог международного нормативного контроля (информации о произведениях и их авторах). В разработке проекта участвовало несколько крупнейших мировых библиотек, в том числе Немецкая национальная библиотека, Библиотека Конгресса США и РНБ.

\begin{figure}
    \centering
    \includegraphics[width=\linewidth]{img/viaf}
    \caption{Сайт VIAF}
    \label{fig:viaf}
\end{figure}

