\chapter{Поля 70x: индивидуальные авторы}

Блок полей 7xx в RUSMARC  -- блок ответственности, он содержит имена лиц и наименования организаций, надел\"eнных той или иной степенью ответственности по отношению к каталогизируемому документу (создание документа, его распространение, владение документом и т.п.).

Ответственность подразделяется на первичную и вторичную. Первичную ответственность могут нести одно лицо, одна организация или один род / династия / семья. Другие лица, роды / династии / семьи, организации, несущие равную с ними ответственность, наделены статусом альтернативной ответственности.

Если основной точкой доступа в записи является заглавие, лица, роды / династии / семьи и организации могут быть наделены статусом альтернативной или вторичной ответственности: авторы (один или несколько авторов с равной степенью ответственности) наделяются статусом альтернативной ответственности; редакторы, переводчики, авторы иллюстраций и т.д. -- статусом вторичной ответственности.

Если невозможно определить уровень ответственности, все лица и организации рассматриваются как несущие вторичную ответственность.

При внесении сведений в данное поле основным моментом является определение фамилии автора или первого элемента при ее отсутствии. Имя автора приводится в форме, получившей наибольшую известность. В качестве ориентира здесь должны выступать энциклопедические словари, справочники и национальный авторитетный файл.

\section{Подполе A: фамилия}

\section{Подполе B: инициалы}

\section{Подполе G: расширение инициалов}

\section{Подполе 9: роль (инвертирование ФИО допустимо?)}

Устанавливается в значение \emph{1} (соответствует положению переключателя \emph{Отменить умолчание}), если имя записано в прямом порядке.

Имя записано в прямом порядке:

\begin{cutelist}
    \item ИМЯ;
    \item ИМЯ ОТЧЕСТВО;
    \item ИМЯ ПРОЗВИЩЕ (ЭПИТЕТ, ОПРЕДЕЛЕНИЕ);
    \item ИМЯ [ФАМИЛИЯ ИМЯ ОТЧЕСТВО] ДУХОВНОЕ ЗВАНИЕ;
    \item ИМЯ ИМЯ ИМЯ (китайские и другие восточные имена);
    \item СОКРАЩЕННОЕ ИМЯ (.) ИНИЦИАЛ;
    \item ИНИЦИАЛ (.) СОКРАЩЕННОЕ ИЛИ ПОЛНОЕ ИМЯ.
\end{cutelist}

Имя записано под фамилией, родовым именем, отчеством:

\begin{cutelist}
    \item ФАМИЛИЯ (,) ИМЯ ОТЧЕСТВО (то же в инициальной форме);
    \item ФАМИЛИЯ (,) ИМЯ (то же инициалы);
    \item ФАМИЛИЯ;
    \item ФАМИЛИЯ (-) ФАМИЛИЯ (,) ИМЯ ОТЧЕСТВО (то же в инициальной форме);
    \item ФАМИЛИЯ ФАМИЛИЯ ФАМИЛИЯ (,) ИМЯ (то же инициалы);
    \item СОКРАЩЕННАЯ ФАМИЛИЯ (,) ИМЯ (то же инициалы).
\end{cutelist}

Если имя приведено в прямом порядке, то в подполях 9 и L должно быть установлено значение \emph{1}, а подполя B и G должны быть пустыми.

Заполнение подполя 9 влияет на автоматическое формирование сведений об ответственности. В частности, если установить его в \emph{1}, то в сформированных сведениях об ответственности инициалы будут помещены после фамилии автора. Поэтому, если на титульном листе ФИО автора приведено как \emph{Иванов А. А.}, то подполе 9 следует установить в \emph{1}.

\section{Подполе 1: неотъемлемая часть имени}

Неотъемлемая часть имени -- та, которая не может измениться в течение жизни. Если часть имени может измениться, то она не является неотъемлемой и заносится в подполе C.

\textbf{Примеры.}

\begin{cutelist}
    \item барон;
    \item младший;
    \item отец;
    \item святитель;
    \item сын;
    \item Jr.
\end{cutelist}

\textbf{Типичные ошибки.}

\begin{cutelist}
    \item диакон;
    \item д-р, профессор Гейдельбергского университета;
    \item оглы;
    \item писатель, общественный деятель.
\end{cutelist}

\section{Подполе C: дополнения к именам, кроме дат}

Любые дополнения к именам (кроме дат), которые не являются неотъемлемой частью имени (титулы, звания, эпитеты, указание должности).

Допускаются сокращения по ГОСТ.

\textbf{Примеры.}

\begin{cutelist}
    \item доктор биологических наук, профессор;
    \item заслуж. учитель РСФСР, канд. пед. наук;
    \item чемпион мира по шахматам (1975-1985).
\end{cutelist}

\textbf{Типичные ошибки.}

\begin{cutelist}
    \item Димитр Христов Чорбаджийский;
    \item Епископ Диоклийский Каллист (Уэр);
    \item лорд;
    \item отец ; французский писатель.
\end{cutelist}

% TODO Шаблоны дополнения

\section{Подполе L: индикатор формы записи имени}

Подполе устанавливается в значение \emph{1} (соответствует положению переключателя \emph{Отменить умолчание}), если в подполе A введена не фамилия, а другой элемент имени.

Если в подполе L установлено значение \emph{1}, то подполя B и G должны быть пустыми.

\section{Подполе D: римские цифры}

Римские цифры, связанные с именами римских пап, членов королевских семей и священнослужителей. Если имеется эпитет (второе имя, прозвище и т.п.), связанный с нумерацией, эпитет также включается в подполе D. При использовании подполя D индикатор -- подполе 9 -- обязательно должен быть пустым.

\textbf{Примеры:}

\begin{cutelist}
    \item \textbf{\^{}A}Петр\textbf{\^{}D}II Карагеоргиевич\textbf{\^{}C}король югославский\textbf{\^{}F}1923-1970\textbf{\^{}9}1
    \item \textbf{\^{}A}Иван\textbf{\^{}C}князь московский\textbf{\^{}D}I Данилович Калита\textbf{\^{}F}ок. 1288-1340
\end{cutelist}

\section{Подполе F: даты жизни}

Даты, присоединяемые к именам лиц, включая слова, указывающие на смысл дат (например, жил, родился, умер). Указанные слова вводятся в подполе в полной или сокращенной форме. Все даты для лица, названного в поле, вводятся в одно подполе F.

Диапазон дат вводится через дефис без пробелов.

Век приводится в числовом виде в арабской нотации. Для веков до нашей эры обязательно указывается "<до н. э.">.

Приблизительные даты приводятся со словом «около».

Даты, вызывающие сомнения (установленные по недостоверным источникам), приводятся со знаком вопроса.

У ныне живущих авторов вводится дата рождения, дефис и пробел.

\textbf{Примеры.}

\begin{cutelist}
    \item 1618-1693
    \item 1797 или 1798-1871
    \item около 1048-после 1122
    \item 1978-
    \item 2 в. до н. э.
    \item 1258-?
\end{cutelist}

\textbf{Типичные ошибки.}

\begin{cutelist}
    \item 712 -- 770
    \item V в. н. э.
    \item 11-й в.
\end{cutelist}

\section{Подполе R: разночтение фамилии}

В подполе R "<Разночтение фамилии"> вводится вместе с инициалами псевдонимы или подлинные имена авторов (в зависимости от того, какое имя было внесено в подполе A "<Фамилия">) выводится в поисковый словарь авторов (наряду с основным именем), и на него готовится ссылочная каталожная карточка.

Применяется одна из тр\"eх схем:

\begin{cutelist}
    \item Имя в прямой форме
    \item Фамилия И. О.
    \item Фамилия, Имя Отчество
\end{cutelist}

\textbf{Примеры.}

\begin{cutelist}
    \item Августин Блаженный
    \item Б. Г.
    \item Ло Хуа-шэн
    \item Низами Гянджеви Абу Мухаммед Ильяс ибн Юсуф
    \item Фон-Визин
    \item Чхартишвили, Григорий Шалвович
\end{cutelist}

\textbf{Типичные ошибки.}

\begin{cutelist}
    \item Екатерина Мечиславовна Насута
    \item Полное имя: Феликс Лопе де Вега и Карпио
    \item Элизабет Роузмонд Тейлор
    \item Эффос, А. Э.
\end{cutelist}

\section{Подполе Y: работает в данной организации}

Переключатель "<Да/Нет">.
Устанавливается в "<Да">, если автор работает в настоящее время (или работал на момент публикации документа) в организации, создающей библиографическое описание.

\section{Подполе P: место работы автора}

Организационно-правовые формы в наименовании организации опускаются.

\textbf{Примеры.}

\begin{cutelist}
    \item Клуб бухгалтеров и аудиторов некоммерческих организаций
    \item Московский государственный университет культуры и искусств
    \item Российский государственный торгово-экономический университет
\end{cutelist}

\textbf{Типичные ошибки.}

\begin{cutelist}
    \item Главный научный сотрудник Отдела взаимосвязей русской и зарубежных литератур Института русской литературы (Пушкинский Дом) РАН
    \item НОУ "Тольяттинская академия управления"
\end{cutelist}

\section{Подполя 4, 5, 6: функция}

Функции лиц со вторичной ответственностью заполнять строго с помощью встроенного меню, не добавляя пояснений вроде "<730 пер. с фр."> После того, как АРМ автоматически сформирует сведения об ответственности, привести их к соответствию с предписанным источником путем редактирования.

Недопустимо вводить две и более функции в одном подполе (например, "<070 авт. и ред.">). Выбирается одна, первая, функция, затем автоматически сформированные сведения об ответственности редактируются вручную для приведения в соответствие с предписанным источником информации.

Соответствие нестандартных функций.

\begin{center}
\begin{tabular}{|l|l|}
    \hline 
    \thead{Нестандартная функция} & \thead{Стандартная функция}  \\ 
    \hline 
    Подготовка текста & 340 редактор  \\ 
    \hline 
    Научная подготовка текста & 340 редактор  \\ 
    \hline 
    Текст &  570 без расшифровки \\ 
    \hline 
    Автор-составитель & 220 составитель \\ 
    \hline 
\end{tabular}
\end{center}

Часто встречаются книги, в которых составитель на титульном листе оформлен как автор, а на обороте титульного листа или в тексте четко обозначено, что он является составителем или переводчиком. В качестве примера: стихи А.~С.~Пушкина в переводе на эвенкийский язык, переводчик В. Кейметинов на титульном листе оформлен как автор (над заглавием). В соответствии с договоренностью между РГБ и РНБ было принято решение следовать за формальными признаками -- оформлением титульного листа. В вышеприведенном примере книга будет оформлена под автором Кейметинов, а Пушкин станет соавтором. В поле 327 "<Примечания о содержании"> будет записано "<Стихи А. С. Пушкина в переводе В. Кейметинова">.
