\chapter{Поля 70x: индивидуальные авторы}

Блок полей 7xx в RUSMARC  -- блок ответственности, он содержит имена лиц и наименования организаций, надел\"eнных той или иной степенью ответственности по отношению к каталогизируемому документу (создание документа, его распространение, владение документом и т.п.).

Ответственность подразделяется на первичную и вторичную. Первичную ответственность могут нести одно лицо, одна организация или один род / династия / семья. Другие лица, роды / династии / семьи, организации, несущие равную с ними ответственность, наделены статусом альтернативной ответственности.

Если основной точкой доступа в записи является заглавие, лица, роды / династии / семьи и организации могут быть наделены статусом альтернативной или вторичной ответственности: авторы (один или несколько авторов с равной степенью ответственности) наделяются статусом альтернативной ответственности; редакторы, переводчики, авторы иллюстраций и т.д. -- статусом вторичной ответственности.

Если невозможно определить уровень ответственности, все лица и организации рассматриваются как несущие вторичную ответственность.

При внесении сведений в данное поле основным моментом является определение фамилии автора или первого элемента при ее отсутствии. Имя автора приводится в форме, получившей наибольшую известность. В качестве ориентира здесь должны выступать энциклопедические словари, справочники и национальный авторитетный файл.

\section{Подполе A: фамилия}

\section{Подполе B: инициалы}

\section{Подполе G: расширение инициалов}

\section{Подполе 9: роль (инвертирование ФИО допустимо?)}

Устанавливается в значение \emph{1} (соответствует положению переключателя \emph{Отменить умолчание}), если имя записано в прямом порядке.

Имя записано в прямом порядке:

\begin{itemize}
    \item ИМЯ;
    \item ИМЯ ОТЧЕСТВО;
    \item ИМЯ ПРОЗВИЩЕ (ЭПИТЕТ, ОПРЕДЕЛЕНИЕ);
    \item ИМЯ [ФАМИЛИЯ ИМЯ ОТЧЕСТВО] ДУХОВНОЕ ЗВАНИЕ;
    \item ИМЯ ИМЯ ИМЯ (китайские и другие восточные имена);
    \item СОКРАЩЕННОЕ ИМЯ (.) ИНИЦИАЛ;
    \item ИНИЦИАЛ (.) СОКРАЩЕННОЕ ИЛИ ПОЛНОЕ ИМЯ.
\end{itemize}

Имя записано под фамилией, родовым именем, отчеством:

\begin{itemize}
    \item ФАМИЛИЯ (,) ИМЯ ОТЧЕСТВО (то же в инициальной форме);
    \item ФАМИЛИЯ (,) ИМЯ (то же инициалы);
    \item ФАМИЛИЯ;
    \item ФАМИЛИЯ (-) ФАМИЛИЯ (,) ИМЯ ОТЧЕСТВО (то же в инициальной форме);
    \item ФАМИЛИЯ ФАМИЛИЯ ФАМИЛИЯ (,) ИМЯ (то же инициалы);
    \item СОКРАЩЕННАЯ ФАМИЛИЯ (,) ИМЯ (то же инициалы).
\end{itemize}

Если имя приведено в прямом порядке, то в подполях 9 и L должно быть установлено значение \emph{1}, а подполя B и G должны быть пустыми.

Заполнение подполя 9 влияет на автоматическое формирование сведений об ответственности. В частности, если установить его в \emph{1}, то в сформированных сведениях об ответственности инициалы будут помещены после фамилии автора. Поэтому, если на титульном листе ФИО автора приведено как \emph{Иванов А. А.}, то подполе 9 следует установить в \emph{1}.

\section{Подполе 1: неотъемлемая часть имени}

Неотъемлемая часть имени -- та, которая не может измениться в течение жизни. Если часть имени может измениться, то она не является неотъемлемой и заносится в подполе C.

\textbf{Примеры.}

\begin{itemize}
    \item барон;
    \item младший;
    \item отец;
    \item святитель;
    \item сын;
    \item Jr.
\end{itemize}

\textbf{Типичные ошибки.}

\begin{itemize}
    \item диакон;
    \item д-р, профессор Гейдельбергского университета;
    \item оглы;
    \item писатель, общественный деятель.
\end{itemize}

\section{Подполе C: дополнения к именам, кроме дат}

Любые дополнения к именам (кроме дат), которые не являются неотъемлемой частью имени (титулы, звания, эпитеты, указание должности).

Допускаются сокращения по ГОСТ.

\textbf{Примеры.}

\begin{itemize}
    \item доктор биологических наук, профессор;
    \item заслуж. учитель РСФСР, канд. пед. наук;
    \item чемпион мира по шахматам (1975-1985).
\end{itemize}

\textbf{Типичные ошибки.}

\begin{itemize}
    \item Димитр Христов Чорбаджийский;
    \item Епископ Диоклийский Каллист (Уэр);
    \item лорд;
    \item отец ; французский писатель.
\end{itemize}

% TODO Шаблоны дополнения

\section{Подполе L: индикатор формы записи имени}

\section{Подполе D: римские цифры}

\section{Подполе F: даты жизни}

\section{Подполе R: разночтение фамилии}

\section{Подполе Y: работает в данной организации}

\section{Подполе P: место работы автора}

\section{Подполя 4, 5, 6: функция}

