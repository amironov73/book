\chapter*{Общие рекомендации по заполнению полей}

\section*{Пробельные символы и знаки препинания}

В элементах БО пробелы должны употребляться в соответствии со следующими требованиями:

\begin{itemize}
	\item Пробелы в начале и в конце элемента недопустимы;
	\item Два и более пробела подряд недопустимы (исключение составляют некоторые подполя полей 330 и 922);
	\item Пробел ставится после знаков препинания (точка, запятая, точка с запятой, вопросительный и восклицательный знаки), но не перед ними. Исключение составляет предписанная пунктуация, вводимая непосредственно в подполе (например, точка с запятой в последующих сведениях об ответственности обрамляется пробелами как спереди, так и сзади);
	\item Пробел ставится перед открывающей скобкой (круглой, квадратной), но не после неё;
	\item Пробел ставится после закрывающей скобки, но не до неё;
	\item Пробел не ставится между двумя соседствующими знаками препинания;
	\item Пробел ставится между числом и обозначением года/века или единицы измерения (например: \textit{2005 г., 100 м});
	\item Пробел ставится внутри инициалов (например: \textit{Миронов А. В.});
	\item Многоточие выделяется пробелом с обеих сторон (например: \textit{всё было б хорошо ... }).
\end{itemize}

\section*{Числа, числительные, римские цифры, единицы измерения}

Числительные во всех служебных полях БО всегда вводятся только в цифровой форме, а не в словесной (прописью).

При записи числительных в цифровой форме необходимо использовать арабские цифры, кроме явно оговорённых случаев (например, наличие в описываемом документе двойной пагинации -- римской и арабской).

Числа записываются без пробелов (даже в тех случаях, когда в описываемом документе применяются пробелы для разбивки длинного числа на группы). Примеры:

\begin{itemize}
	\item \textit{20 000 лье под водой} $\rightarrow$ \textit{20000 лье под водой};
	\item \textit{8 201 794} $\rightarrow$ \textit{8201794}.
\end{itemize}

\section*{Форма написания дат и интервалов времени}

Написание дат и интервалов времени регулируется ГОСТ 7.64-90 "Представление дат и времени дня. Общие требования". Также о написании дат и интервалов времени см. "Справочник издателя и автора", глава 7, особенно раздел 7.2.

В элементах БО, таких как хронологические рубрики, подрубрики и аналогичных по назначению, обозначение единицы измерения интервала времени (года, века и т. д.) всегда отделяется от числа пробелом. Применяется сокращённая форма написания единицы с точкой на конце.

Правильно: \textit{2005 г., 1900-1950 гг., 19-20 вв.}

Неправильно: \textit{2005г., 1900-50гг., XIX в., XIX-XX вв.}

Правила написания интервалов времени см п. 4.4.3 РПК.

Правила написания дат и интервалов времени в элементах БО, таких как заглавие, см. в соответствующих разделах.
