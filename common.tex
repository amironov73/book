\chapter{Общие рекомендации по заполнению полей}

\section{Пробельные символы и знаки препинания}

В элементах БО пробелы должны употребляться в соответствии со следующими требованиями:

\begin{itemize}
	\item Пробелы в начале и в конце элемента недопустимы;
	\item Два и более пробела подряд недопустимы (исключение составляют некоторые подполя полей 330 и 922);
	\item Пробел ставится после знаков препинания (точка, запятая, точка с запятой, вопросительный и восклицательный знаки), но не перед ними. Исключение составляет предписанная пунктуация, вводимая непосредственно в подполе (например, точка с запятой в последующих сведениях об ответственности обрамляется пробелами как спереди, так и сзади);
	\item Пробел ставится перед открывающей скобкой (круглой, квадратной), но не после неё;
	\item Пробел ставится после закрывающей скобки, но не до неё;
	\item Пробел не ставится между двумя соседствующими знаками препинания;
	\item Пробел ставится между числом и обозначением года/века или единицы измерения (например: \textit{2005 г., 100 м});
	\item Пробел ставится внутри инициалов (например: \emph{Миронов~А.~В.});
	\item Многоточие выделяется пробелом с обеих сторон (например: \emph{всё было б хорошо \dots }).
\end{itemize}

\section{Числа, числительные, римские цифры, единицы измерения}

Числительные во всех служебных полях БО всегда вводятся только в цифровой форме, а не в словесной (прописью).

При записи числительных в цифровой форме необходимо использовать арабские цифры, кроме явно оговор\"eнных случаев (например, наличие в описываемом документе двойной пагинации -- римской и арабской).

Числа записываются без пробелов (даже в тех случаях, когда в описываемом документе применяются пробелы для разбивки длинного числа на группы). Примеры:

\begin{itemize}
	\item \emph{20 000 лье под водой} $\rightarrow$ \emph{20000 лье под водой};
	\item \emph{8 201 794} $\rightarrow$ \emph{8201794}.
\end{itemize}

\textbf{Номер тома, выпуска, части} и~т.~п. всегда пишется только в цифровой форме в арабской нотации.

Правильно: \emph{Т. 1, Ч. 2, Вып. 3, Кн. 4.}

Неправильно: emph{Т II, Часть третья.}

\textbf{Тире в числовых интервалах.} Тире, обозначающее интервал значений, выделяют пробелами при словесной или смешанной форме чисел, например: \emph{пять -- семь минут}, \emph{20 тыс. -- 30 тыс.}, но приводят без пробелов при цифровой форме чисел, например: \emph{1941--1945}, \emph{с. 12--45}, \emph{разд. 6--11} (РПК, п. 4.4.3).

Поскольку символ тире отсутствует на клавиатуре, он заменяется дефисом (минусом).

\textbf{Римские цифры}. Любые числа, записанные в римской нотации (например, \emph{XXI}, \emph{MCMLXXIII} и~т. ~п.), должны вводиться строго лишь в латинской раскладке клавиатуры. Заменять латинские буквы \emph{X}, \emph{C} сходными по написанию кириллическими \emph{Х} и \emph{С} недопустимо! Также нельзя заменять букву \emph{I} на цифру \emph{1}. Вообще, при написании чисел в римской нотации допустимо использовать лишь прописные латинские буквы \emph{C}, \emph{I}, \emph{L}, \emph{M}, \emph{V} и \emph{X}.

В поле 10 "<Цена, ISBN"> последнюю (контрольную) цифру ISBN также вводят только в латинской раскладке.

\textbf{Единицы измерения величин}, как правило, вводятся без точки (ГОСТ 8.417-2002), однако, следует иметь в виду, что в обозначениях момента времени точка используется (Справочник издателя и автора, раздел 7.2, см. также ниже).

\smallskip
\noindent\begin{tabularx}{\linewidth}{|X|X|}
    \hline 
    \thead{Вводятся без точки} & \thead{Вводятся с точкой}  \\ 
    \hline 
    час (ч), минута (мин), секунда (сек) в обозначении интервала времени &  год (г.), месяц (мес.), час (ч.), минута (мин.) в обозначениях момента времени \\ 
    \hline 
    кило-, санти-, деци-, милли-, микро- метр (м), грамм (г), литр (л), градус (град)  & \\ 
    \hline 
    тонна (т), ценнер (ц)  &  \\ 
    \hline 
    ньютон (Н), ампер (А), ом (Ом), вольт (В), ватт (Вт) &  \\ 
    \hline 
\end{tabularx} 

\section{Форма написания дат и интервалов времени}

Написание дат и интервалов времени регулируется ГОСТ 7.64-90 "<Представление дат и времени дня. Общие требования">. Также о написании дат и интервалов времени см. "<Справочник издателя и автора">, глава 7, особенно раздел 7.2.

В элементах БО, таких как хронологические рубрики, подрубрики и аналогичных по назначению, обозначение единицы измерения интервала времени (года, века и т. д.) всегда отделяется от числа пробелом. Применяется сокращённая форма написания единицы с точкой на конце.

Правильно: \emph{2005 г., 1900-1950 гг., 19-20 вв.}

Неправильно: \emph{2005г., 1900-50гг., XIX в., XIX-XX вв.}

Правила написания интервалов времени см п. 4.4.3 РПК.

Правила написания дат и интервалов времени в элементах БО, таких как заглавие, см. в соответствующих разделах.

\section{Наращение окончаний числительных}

Наращение окончаний допустимо только для порядковых числительных, записанных арабскими цифрами. У количественных числительных окончание не наращивается никогда!

Наращение не используется:

\begin{cutelist}
    \item при записи календарных чисел: \emph{22 марта};
    \item если число обозначено римскими цифрами: \emph{IX конгресс};
    \item в номерах томов, глав, страниц, иллюстраций, таблиц, приложений, если родовое слово (том, глава) предшествует числительному: \emph{на с. 196}, \emph{в т. 5} (но \emph{на 196-й странице}, в \emph{5-м томе}).
\end{cutelist}

Наращение всегда пишется с дефисом (т. е. пробелы между числом и окончанием недопустимы). Правильно: \emph{2-е издание}. Неправильно: \emph{2~-е издание}.

Наращение падежного окончания в порядковых числительных, обозначенных арабскими цифрами, может быть однобуквенным или двухбуквенным.

По закрепившейся традиции наращение должно быть однобуквенным, если последней букве числительного предшествует гласный звук: \emph{5-й день} (пятый день), \emph{25-я годовщина} (двадцать пятая годовщина), \emph{в 32-м издании} (в тридцать втором издании), \emph{в 14-м} ряду (в четырнадцатом ряду).

Наращение должно быть двубуквенным, если последней букве предшествует согласный: \emph{5-го дня} (пятого дня), \emph{к 25-му студенту} (к двадцать пятому студенту), из \emph{32-го издания} (из тридцать второго издания), \emph{из 14-го ряда} (из четырнадцатого ряда).

Если подряд следуют два порядковых числительных, раздел\"eнных запятой или соедин\"eнных союзом, падежное окончание наращивают у каждого из них: \emph{1-й, 2-й вагоны}; \emph{80-е и 90-е годы}.

Если подряд следуют более двух порядковых числительных, раздел\"eнных запятой, точкой с запятой или соедин\"eнных союзом, то падежное окончание наращивают только у последнего числительного: \emph{1, 2 и 3-й вагоны}, \emph{70, 80, 90-е годы}.

Если два порядковых числительных следуют через тире, то падежное окончание наращивают:

\begin{itemize}
    \item[а)] только у второго числительного, если падежное окончание у обоих числительных одинаковое: \emph{50–60-е годы}, \emph{в 80–90-х годах};
    \item [б)] у каждого числительного, если падежные окончания разные: \emph{в 11-м – 20-х рядах}.
\end{itemize}

% Без наращения приводятся порядковые номера томов, глав, страниц, классов, курсов, если соответствующие родовые слова (том, глава и т. д.) предшествуют порядковому числительному.

\section{Кавычки и апострофы}

Во всех полях можно употреблять только машинописные двойные кавычки '', часто называемые "<лапками">, даже если в описываемом документе применяется другой тип кавычек.

Это связано с тем, что ИРБИС неверно обрабатывает любые другие кавычки. Апострофы ' и одинарные кавычки ‘ заменять на двойные кавычки не надо. Обратный апостроф ` в БО не используется.

% TODO Таблица

\section{Сокращение слов}

C 1 июля 2019 года вступил в силу ГОСТ Р 7.0.100-2018 "<Библиографическая запись. Библиографическое описание. Общие требования и правила составления">. В н\"eм значительно сокращается употребление сокращений в описаниях. 

% TODO далее

\section{Употребление букв Е и \"E}

Правила употребления буквы \"E приведены в академическом справочнике "<Правила русской орфографии и пунктуации"> под редакцией В.~В.~Лопатина. В частности, \S{} 5 утверждает следующее.

\begin{quotation}
    Употребление буквы \"e может быть последовательным и~выборочным.
    
    Последовательное употребление буквы \"e обязательно в~следующих разновидностях печатных текстов:
    
    а) в текстах с последовательно поставленными знаками ударения;
    
    б) в книгах, адресованных детям младшего возраста;
    
    в) в учебных текстах для школьников младших классов и иностранцев, изучающих русский язык.
    
    В обычных печатных текстах буква \"e употребляется выборочно. Рекомендуется употреблять ее в следующих случаях.
    
    1. Для предупреждения неправильного опознания слова, напримеп: \emph{вс\"e}, \emph{н\"eбо}, \emph{л\"eтом}, \emph{соверш\"eнный} (в отличие соответственно от слов \emph{все}, \emph{небо}, \emph{летом}, \emph{совершенный}), в том числе для указания на место ударения в слове, например: \emph{в\"eдро}, \emph{узна\"eм} (в отличие от \emph{ведро}, \emph{узнаем}).
    
    2. Для указания правильного произношения слова — либо редкого, недостаточно хорошо известного, либо имеющего распростран\"eнное неправильное произношение, например: \emph{г\"eзы}, \emph{с\"eрфинг}, \emph{фл\"eр}, \emph{тв\"eрже}, \emph{щ\"eлочка}, в том числе для указания правильного ударения, например: \emph{побас\"eнка}, \emph{привед\"eнный}, \emph{унес\"eнный}, \emph{осужд\"eнный}, \emph{новорожд\"eнный}, \emph{фил\"eр}.
    
    3. В собственных именах -- фамилиях, географических названиях, например: \emph{Кон\"eнков}, \emph{Не\"eлова}, \emph{Катрин Ден\"eв}, \emph{Шр\"eдингер}, \emph{Дежн\"eв}, \emph{Кошел\"eв}, \emph{Чебыш\"eв}, \emph{В\"eшенская}, \emph{Ол\"eкма}.
\end{quotation}

Таким образом, в электронном каталоге употребление буквы \"E выборочное. При заполнении полей всюду, где это возможно, букву \"E необходимо заменять на Е.