\chapter{Поле 200: основное заглавие}

Блок 2xx в RUSMARC -- блок описательной информации, он содержит данные для всех областей описания ISBD, кроме Области примечания и Области стандартного номера (или его альтернативы) и условий доступности стандартов ISBD и ГОСТ 7.1-2003. 

Согласно стандарту RUSMARC

\begin{quotation}
	Элементы данных в блоке 2{-}{-} не должны генерироваться автоматически из элементов данных полей других блоков исходной записи при отсутствии их в блоке. Например, если в исходной записи отсутствует элемент данных "<сведения об ответственности">, поля блока 7{-}{-} (сведения об ответственности в форме точки доступа) не должны использоваться для генерации этого элемента, так как нет гарантии, что данные в таком сгенерированном поле будут добавлять что-либо к описательной части записи.
\end{quotation}

Однако АРМ "<Каталогизатор"> нарушает это требование, и сведения об ответственности генерируются автоматически, если они не заполнены вручную.

Впрочем, это можно отключить в настройках АРМ.

Поле 200 в ИРБИС соответствует полю RUSMARC: 200

ГОСТ 7.1-2003, пункт 5.2.2.

Поле обязательное, не повторяющееся.

\section{Подполе A: основное заглавие}

Приводится полностью в том виде и в той последовательности, как оно дано в публикации, с сохранением имеющихся знаков препинания.

При отсутствии в заглавии знаков препинания между фразами они отделяются друг от друга точкой.

\textbf{Примеры:}

\begin{itemize}
	\item Инновации? Инновации… Инновации!
	\item Культура. Информация. Технология
	\item Десять лет, которые потрясли … 1991-2001
\end{itemize}

Основное заглавие может содержать альтернативное заглавие, соединенное с ним союзом "<или"> и записываемое с прописной буквы. Перед союзом «или» ставят запятую. Альтернативное заглавие повторяется в подполе 517\^{}а.

\textbf{Примеры:}

\begin{itemize}
	\item Come undone, или Макабрические миры доктора Делюмо
	\item 47 принципов древних самураев, или Кодекс руководителя
	\item Августейший сезон, или Книга российских календ
\end{itemize}

При наличии в начале заглавия инициалов или раскрытых инициалов с фамилией лица, то заглавие без инициалов или раскрытых инициалов отражается дополнительно в подполе 517\^{}а.

\textbf{Примеры:}

\begin{itemize}
	\item Владимир Владимирович Маяковский -- человек и поэт
	\item А. С. Макаренко о воспитании детей в семье
	\item Александр Вампилов в воспоминаниях и фотографиях
\end{itemize}

Если основное заглавие начинается с цифрового обозначения, то в подполе 517\^{}а повторяются варианты заглавия:

\begin{itemize}
	\item[а)] цифра дается в словесном выражении и наоборот;
	\item[б)] римская цифра заменяется арабской и наоборот;
	\item[в)] без цифрового обозначения (для чтений, посвященных лицу).
\end{itemize}

\textbf{Пример:}

\begin{itemize}
	\item 200\^{}а – Х Международная научная конференция документоведов
	\item 517\^{}а – 10 Международная научная конференция документоведов
	\item 517\^{}а – Десятая Международная научная конференция документоведов
\end{itemize}

Если заглавие состоит из двух и более значимых фраз, то в поле 517 создается точка доступа на все фразы, кроме первой, при этом тип разночтения устанавливается "<517 Вариант заглавия">. В подполе "<Роль (основная карточка)"> ставится переключатель "<Нет">.

\textbf{Примеры.}

\begin{itemize}
	\item Автоматическое регулирование. Теория и элементы управляющих систем
	\item Беседы о русской культуре. Быт и традиции русского дворянства
	\item Введение в изучение права и нравственности. Эмоциональная психология 
\end{itemize}

Если в предписанном источнике информации типовые (нехарактерные) заглавия приведены через точку, например, "<Романы. Повести. Рассказы"> или в отдельных строках, такое заглавие вводится целиком в 200\^{}a плюс создается дополнительная точка доступа на каждое заглавие, стоящее после точки ("<Повести"> и "<Рассказы"> в вышеприведенном примере).

Если часть заглавия не представляет собой самостоятельно значимой фразы, она не повторяется в поле 517.

\textbf{Примеры.}

\begin{itemize}
	\item Философия современной России. Какая она?
	\item Мы не пойдем на Север. Почему?
\end{itemize}

При отсутствии в публикации заглавия, его формулируют и приводят в квадратных скобках.

Недопустима формулировка "<Без заглавия">.

В конце заглавия точка не ставится, кроме случаев, когда оно заканчивается многоточием.

В заглавии сокращения слов не допускаются, кроме случаев, когда слова сокращены автором.

% TODO орфография и далее

\section{Подполе B: общее обозначение материала}

{\color{red}В новом ГОСТ больше не используется}

ГОСТ 7.1-2003, пункт 5.2.3

Указывает на носитель информации, содержащего каталогизируемый документ.

\begin{tabular}{|l|l|}
	\hline 
	\thead{На русском языке}& \thead{На иностранном языке}  \\ 
	\hline 
	Видеозапись &  Videorecording \\ 
	\hline 
	Звукозапись &  Sound recording \\ 
	\hline 
	Изоматериал &  Graphic \\ 
	\hline 
	Карты &  Cartographic material \\ 
	\hline 
	Комплект &  Kit \\ 
	\hline 
	Кинофильм &  Motion picture \\ 
	\hline 
	Микроформа &  Microform \\ 
	\hline 
	Мультимедиа &  Multimedia \\ 
	\hline 
	Ноты &  Music \\ 
	\hline 
	Предмет &  Object \\ 
	\hline 
	Рукопись &  Manuscript \\ 
	\hline 
	Текст &  Text \\ 
	\hline 
	Шрифт Брайля &  Braille \\ 
	\hline 
	Электронный ресурс & Electronic resource \\ 
	\hline 
\end{tabular} 

% TODO мультимедиа и проч

\section{Подполе E: сведения, относящиеся к заглавию}

ГОСТ 7.1-2003, пункт 5.2.5

Сведения в подполе раскрывают и поясняют заглавие произведения. Здесь также приводятся сведения о литературном жанре произведения, указание о том, что оно является переводом с другого языка (при отсутствии фамилии переводчика) и т. п.

Сведения указываются в том порядке и в том виде, как они приведены в публикации. Если сведений в предписанном источнике информации нет, их можно взять из другого источника или сформулировать самостоятельно. В этом случае сведения приводятся в квадратных скобках. Типичные сведения для печатных изданий:

% TODO альманах и т. д.

\section{Подполя F: первые сведения об ответственности и G: последующие сведения об ответственности}

ГОСТ 7.1-2003, пункт 5.2.6

Сведения об ответственности содержат информацию о лицах и организациях, участвовавших в создании интеллектуального, художественного или иного содержания произведения, являющегося объектом описания. Сведения об ответственности записывают в той форме, в какой они указаны в предписанном источнике информации.

Сведения приводятся со строчной буквы, кроме случаев, когда они начинаются с имен собственных или аббревиатуры, начинающейся с прописной буквы.

Сокращение отдельных слов и словосочетаний производится согласно ГОСТ.
Для произведений, имеющих одного, двух и трех авторов, приводятся сведения об этих авторах в том виде, как они указаны на титульном листе.

Для произведений, имеющих четырех и более авторов, приводятся сведения о первом авторе в том виде, как он указан в статье, с добавлением в квадратных скобках слов [и др.].

Если сведения об авторе являются составной частью заглавия, то в первых сведениях об ответственности они не приводятся. Примеры: «Дневник А. И. Деникина», «Письма А. А. Ахматовой».

Никогда не отражаются в подполе 200\^{}f авторы произведений из подборок, рецензируемых книг, статей и авторы выступлений на мероприятиях. Также никогда не отражаются авторы цитат.

В ИРБИС предусмотрено три способа заполнения подполя:

\begin{itemize}
	\item простой ввод с клавиатуры -- самый гибкий, но одновременно и самый трудоёмкий способ;
	\item с использованием словаря авторов, когда система подставляет в подполе список отобранных авторов, разделённых запятыми, если необходимо, автоматически выполняя инверсию ФИО. Временные и постоянные коллективные авторы разделяются точкой с запятой;
	\item если подполе не заполнено на момент сохранения записи, оно автоматически формируется из введенных сведений об авторах и лицах со вторичной ответственностью. При этом, если необходимо, выполняется инверсия ФИО. Автоматическое формирование подполя можно отключить в настройках АРМ, если в настройке (по кнопке или в поле 905 на странице Технология) задать "<нет"> для признака "<Формировать Сведения об ответственности?"> (по умолчанию сведения об ответственности формируются автоматически).
\end{itemize}

% TODO предпочтительный порядок

\section{Подполе U: роль (нехарактерное заглавие, доб. карточка?)}

Согласно "<Издательскому словарю">

\begin{quotation}
	ХАРАКТЕРНОЕ ЗАГЛАВИЕ -- вид тематического заглавия, словесно выражающего инд. тему, идею, образ, цель, адрес издания, в отличие от заглавия типового (нехарактерного).
	
	ТИПОВОЕ ЗАГЛАВИЕ -- тематическое заглавие издания, определяющее лишь его жанр, вид, назначение и в ряде случаев неотличимое от таких же заглавий других изданий. Напр.: Путеводитель, Каталог, Бюллетень, Труды, Науч. труды.	
\end{quotation}

% TODO кроме того

\section{Подполе V: обозначение и номер тома}

Подполе вводится в одном из двух форматов.

1. Сокращ\"eнное обозначение части многотомного издания, пробел и порядковый номер части.

{\noindent\scriptsize\begin{tabular}{|l|l|l|l|}
	\hline 
	\thead{Обозначение части} & \thead{Сокращение}  & \thead{Обозначение части}  & \thead{Сокращение}  \\ 
	\hline 
	\multicolumn{2}{|c|}{для русскоязычных документов} &  \multicolumn{2}{|c|}{для документов на иностранных языках}  \\ 
	\hline 
	Альбом & Альб. & Album & Alb.  \\ 
	\hline 
	Выпуск &  Вып. &  Issue & Iss.  \\ 
	\hline 
	Диск &  Диск &  Disk & Disk  \\ 
	\hline 
	Дополнительный выпуск &  Доп. вып. & Additional issue & Add. iss.  \\ 
	\hline 
	Дополнительный том &  Доп. т. &  Additional volume & Add. vol.  \\ 
	\hline 
	Кассета & Кас. &  Cassette & Cas. \\ 
	\hline 
	Книга & Кн. & Book & Book \\ 
	\hline 
	Сборник &  Сб. & Collection & Col.  \\ 
	\hline 
	Спецвыпуск & Спецвып. & Special issue & Spec. iss. \\ 
	\hline 
	Специальный выпуск &  Спец. вып. & Special issue & Spec. iss. \\ 
	\hline 
	Том &  Т. & Volume & Vol. \\ 
	\hline 
	Часть &  Ч. & Part  & Pt.  \\ 
	\hline 
\end{tabular} 
}