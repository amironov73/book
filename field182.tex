\chapter{Поле 182: коды - тип средства}

Поле содержит кодированные данные, определяющие средство доступа, которое характеризует возможности хранения, использования или передачи содержания каталогизируемого ресурса как с помощью специализированных устройств (аппаратов), так и без них.

Поле введено в ИРБИС64+ версии 2018.1 в связи с переходом на ГОСТ 7.0.100-2018.

Поле факультативное. Повторяется, если использование ресурса возможно с помощью различных средств доступа, либо используется более одной системы кодов.

\section{Подполе A: код типа средства}

Код средства доступа (ISBD / ГОСТ Р 7.0.100–2018)
Одна позиция символа определяет средство доступа, в соответствии с определениями ISBD / ГОСТ Р 7.0.100–2018 для этого элемента.

Факультативное. Не повторяется.

Значения берутся из справочника \emph{182hs.mnu}.

\begin{cutelist}
    \item \textbf{a} -- аудио
    \item \textbf{b} -- электронное
    \item \textbf{c} -- микроформа
    \item \textbf{d} -- микроскопическое
    \item \textbf{e} -- проекционное
    \item \textbf{f} -- стереографическое
    \item \textbf{g} -- видео
    \item \textbf{m} -- разные средства
    \item \textbf{n} -- непосредственное
    \item \textbf{z} -- другое средство
\end{cutelist}

При необходимости данные о средстве доступа могут вводиться в текстовой форме в поле 203. В таком случае поле 203 рекомендуется использовать в дополнение к полю 182 (а не вместо него); индикатор 2 в поле 182 при этом должен иметь значение 0.

Коды в поле 182 могут не соответствовать текстовым данным в поле 203. Порядок генерации вывода определяется в соответствии с практикой учреждения, использующего записи.

Формат допускает использование полей 181 и 182 для автоматической генерации Области 0 ISBD и Области вида содержания и средства доступа ГОСТ Р 7.0.100-2018. Однако, в соответствии с ГОСТ Р 7.0.100-2018, термины характеристики содержания приводят в грамматическом согласовании с терминами, обозначающими вид содержания, поэтому автоматическая генерация Области вида содержания и средства доступа возможна только при условии реализации в информационной системе необходимых механизмов морфологического анализа. В связи с этим рекомендуется для формирования дисплейного представления включать в запись поле 203, а в полях 181 и 182 указывать Инд.2=0.
