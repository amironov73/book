\chapter*{Поле 10: ISBN, цена}

ГОСТ 7.1-2003, п. 5.9

Поле повторяется для нескольких ISBN одного издания.
 
Вопросы, связанные с Международным стандартным книжным номером (ISBN) регулируются ГОСТ 7.0.53-2007.

ISBN действующего стандарта состоит из тринадцати цифр, поделенных на пять групп, разделяемых знаком дефиса.

1.	Префикс EAN.UCC -- код 978 (в дальнейшем будет использоваться 979), предоставленный Европейской ассоциацией товарной нумерации (EAN) Международному агентству ISBN для обозначения товара "Книжная продукция".

2.	Номер регистрационной группы служит для обозначения в ISBN страны, географической или языковой области. Для Российской Федерации номер регистрационной группы -- цифра 5.

3.	Номер регистранта идентифицирует в системе ISBN конкретного издателя, производителя документов. Номер регистранта российский издатель (производитель документов) получает в Российском национальном агентстве ISBN, функционирующем в составе Российской книжной палаты.

4.	Номер издания (публикации) идентифицирует конкретное издание (публикацию) издателя, производителя документов в предоставленном ISBN.

5.	Контрольная цифра служит для проверки правильности цифровой части ISBN.

ISBN старого стандарта состоял из десяти цифр, поделенных на четыре груп-пы, – в нем не было префикса EAN, остальные группы совпадают с действующим стандартом.

\textbf{Примеры.}

\begin{itemize}
	\item 0-812-57558-X \textit{(старый стандарт)};
	\item 5-02-003157-7;
	\item 978-5-98846-049-7 \textit{(действующий стандарт)}.
\end{itemize}

В издании, выпущенном совместно несколькими издателями (в том числе, российскими и зарубежными издателями) приводят ISBN каждого издателя-партнера. Наименование издателя указывают после соответствующего ISBN в круглых скобках без кавычек в той форме, как оно приведено на титульной странице.

\textbf{Примеры.}

\begin{itemize}
	\item 978-5-09-014485-8 (Просвещение);
	\item 978-5-472-01012-2 (Экзамен);
	\item 978-1 -84334-151-2 (Chandos Publishing);
	\item 978-5-93913-059-3 (Профессия).
\end{itemize}

В томе (выпуске) многотомного издания приводят ISBN данного тома (с указанием в круглых скобках обозначения и номера тома) и ISBN многотомного издания в целом.

\textbf{Пример.}

\begin{itemize}
	\item ISBN 978-5-02-033899-9 (т. 1);
	\item ISBN 978-5-02-033897-5.
\end{itemize}

В издании, входящем в состав комплектного, комбинированного издания, приводят ISBN данного издания (с указанием в круглых скобках сведений «отд. кн.» или «отд. изд.») и ISBN комплектного, комбинированного издания в целом.

\textbf{Пример.}

\begin{itemize}
	\item 978-5-89349-822-6 (отд. кн.)
	\item 978-5-89349-820-2
\end{itemize}

ISBN комплектного комбинированного издания в целом приводят на футляре, папке, обложке комплектного издания.

В поле вводятся только ISBN, относящиеся непосредственно к обрабатываемому документу. Прочие ISBN, относящиеся, например, к электронной версии или к изданию в твердом переплете, опускаются. Также опускается ISBN, относящийся к многочастному документу в целом.
Если невозможно определить, какой ISBN к чему относится, то приводят все имеющиеся ISBN (ГОСТ 7.1-2003, п. 5.9.2).
