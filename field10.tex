\chapter{Поле 10: ISBN, цена}

ГОСТ 7.1-2003, п. 5.9 % TODO Ссылка на новый ГОСТ

Поле повторяется для нескольких ISBN одного издания.
 
Вопросы, связанные с Международным стандартным книжным номером (ISBN) регулируются ГОСТ 7.0.53-2007.

ISBN действующего стандарта состоит из тринадцати цифр, поделенных на пять групп, разделяемых знаком дефиса.

\begin{enumerate}
    \item Префикс EAN.UCC -- код \emph{978} (в дальнейшем будет использоваться \emph{979}), предоставленный Европейской ассоциацией товарной нумерации (EAN) Международному агентству ISBN для обозначения товара "<Книжная продукция">.
    \item Номер регистрационной группы служит для обозначения в ISBN страны, географической или языковой области. Для Российской Федерации номер регистрационной группы -- цифра \emph{5}.
    \item Номер регистранта идентифицирует в системе ISBN конкретного издателя, производителя документов. Номер регистранта российский издатель (производитель документов) получает в Российском национальном агентстве ISBN, функционирующем в составе Российской книжной палаты.
    \item Номер издания (публикации) идентифицирует конкретное издание (публикацию) издателя, производителя документов в предоставленном ISBN.
    \item Контрольная цифра служит для проверки правильности цифровой части ISBN.
\end{enumerate}

ISBN старого стандарта состоял из десяти цифр, поделенных на четыре группы, -- в нем не было префикса EAN, остальные группы совпадают с действующим стандартом.

\textbf{Примеры.}

\begin{cutelist}
	\item 0-812-57558-X \emph{(старый стандарт)};
	\item 5-02-003157-7;
	\item 978-5-98846-049-7 \emph{(действующий стандарт)}.
\end{cutelist}

В издании, выпущенном совместно несколькими издателями (в том числе, российскими и зарубежными издателями) приводят ISBN каждого издателя-партнера. Наименование издателя указывают после соответствующего ISBN в круглых скобках без кавычек в той форме, как оно приведено на титульной странице.

\textbf{Примеры.}

\begin{cutelist}
	\item 978-5-09-014485-8 (Просвещение);
	\item 978-5-472-01012-2 (Экзамен);
	\item 978-1 -84334-151-2 (Chandos Publishing);
	\item 978-5-93913-059-3 (Профессия).
\end{cutelist}

В томе (выпуске) многотомного издания приводят ISBN данного тома (с указанием в круглых скобках обозначения и номера тома) и ISBN многотомного издания в целом.

\textbf{Пример.}

\begin{cutelist}
	\item ISBN 978-5-02-033899-9 (т. 1);
	\item ISBN 978-5-02-033897-5.
\end{cutelist}

В издании, входящем в состав комплектного, комбинированного издания, приводят ISBN данного издания (с указанием в круглых скобках сведений "<отд. кн."> или "<отд. изд.">) и ISBN комплектного, комбинированного издания в целом.

\textbf{Пример.}

\begin{cutelist}
	\item 978-5-89349-822-6 (отд. кн.)
	\item 978-5-89349-820-2
\end{cutelist}

ISBN комплектного комбинированного издания в целом приводят на футляре, папке, обложке комплектного издания.

В поле вводятся только ISBN, относящиеся непосредственно к обрабатываемому документу. Прочие ISBN, относящиеся, например, к электронной версии или к изданию в твердом переплете, опускаются. Также опускается ISBN, относящийся к многочастному документу в целом.
Если невозможно определить, какой ISBN к чему относится, то приводят все имеющиеся ISBN (ГОСТ 7.1-2003, п. 5.9.2). % TODO ссылка на новый ГОСТ

\section{Подполе A: ISBN}

Подполе имеет ФЛК, проверяющее правильность написания ISBN по контрольной цифре и сверяющее его на дублетность. Контроль не блокирующий, при наличии ошибок можно продолжать ввод и сохранять запись.

\textbf{Типичные ошибки.}

\begin{cutelist}
    \item употребление кириллической буквы Х вместо латинской X;
    \item употребление пробелов вместо дефисов в качестве разделителя групп;
    \item запись ISBN без разделителей;
    \item ввод уточняющих сведений, которые должны быть внесены в подполе B.
\end{cutelist}

\section{Подполе B: уточнения}

\textbf{Примеры.}

\begin{cutelist}
    \item Бином
    \item Кн. 1
    \item в пер.
\end{cutelist}

Типичная ошибка: ввод сведений в скобках.

\section{Подполе Z: ошибочный ISBN}

Подполе Z отличается от подполя A отсутствием ФЛК. Сюда нужно вносить ISBN, про которые точно известно, что они ошибочные.

\section{Подполе D: общая для всех экземпляров цена}

\textbf{Примеры.}

\begin{cutelist}
    \item 1.00
    \item 1000.00
    \item 123.45
\end{cutelist}

\textbf{Типичные ошибки.}

\begin{cutelist}
    \item 5 руб.
    \item 10 коп.
    \item 10,50
    \item 12
\end{cutelist}

\section{Подполе C: обозначение валюты}

\textbf{Примеры.}

\begin{cutelist}
    \item Fr
    \item USD
    \item тенге
\end{cutelist}

\textbf{Типичные ошибки.}

\begin{cutelist}
    \item руб.
    \item \$
\end{cutelist}
