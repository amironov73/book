\chapter*{Поле 300: общие примечания}

Соответствует полю RUSMARC: 300

ГОСТ 7.1-2003, п. 5.8

Повторяется. Если вводится более одного примечания, каждое вводится в новое повторение поля.

Содержит дополнительную информацию об объекте описания, которая не была приведена в других элементах описания. Сведения, приводимые в поле, заимствуют из любого источника и в квадратные скобки не заключают.

Слова текста, заключенные в угловые скобки < >,  включаются в словарь ключевых слов.

\textbf{Примеры.}

\begin{itemize}
	\item В надзаг.: Посвящается 60-летию Уфимского государственного нефтяного технического университета
	\item Деп. в ВИНИТИ 18.05.02, № 14432
	\item Издание выходило полутомами с последовательной нумерацией выпусков на корешках переплета
	\item Лауреат конкурса «Профессиональный учебник»
	\item На корешке указан том серии
	\item На шмуцтитуле: «Издано под наблюдением Комиссии при Комитете состоящего под высочайшим государя императора покровительством Императорского Общества любителей древней письменности»
	\item Посвящается памяти академика Д. С. Белянкина
	\item Продолжение романа «Ветер над полем»
	\item Произведение печатается без сокращений
	\item Электронные версии книг на сайте www.prospeckt.org
\end{itemize}

\textbf{Типичные ошибки.}

\begin{itemize}
	\item (Библиотека детского романа)
	\item (На обл: Опыт передового учителя)
	\item Авторы указаны на обороте тит. л. - Библиогр.: с. 287 (9 назв.).
	\item Загл. корешка: Великий Октябрь и реальный социализм
	\item Часть текста: англ.
\end{itemize}

