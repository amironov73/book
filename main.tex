\documentclass[a5paper,openany,10pt]{book}
\usepackage[utf8]{inputenc}
\usepackage[english,russian]{babel}
\usepackage[inner=20mm, outer=15mm, top=20mm, bottom=20mm]{geometry}
\usepackage{titlesec} 
\usepackage{fancyhdr} 
\usepackage{tabularx}
\usepackage{colortbl}
\usepackage{xcolor}
\usepackage{graphicx}
\usepackage{tocloft}

% нумерация внизу по центру
\fancyhf{}
\renewcommand{\headrulewidth}{0pt}
\chead{\textit{\chaptertitlename}}
\cfoot{\thepage}
\pagestyle{fancy}   

\newcommand{\thead}[1]{\cellcolor{black}\color{white}\textbf{#1}}	

\begin{document}
	
\frontmatter
\author{Алексей Миронов}
\title{Практика каталогизации в ИРБИС64: методические материалы с уч\"eтом ГОСТ 7.0.100-2018}
\date{Июнь 2019}

\begin{titlepage}
\clearpage\vspace*{\fill}
\begin{center}
	{\Large А. В. Миронов}
	
	\vspace{55mm}
	
	{\sffamily
		{\Huge\textbf{\textsc{Практика каталогизации\\в ИРБИС64}}}
	
		\vspace{5mm}
	
		{\large \textbf{методические материалы \\с уч\"eтом ГОСТ 7.0.100-2018}}
    }

	\vspace{65mm}
	
	{\large Иркутск, 2019}
\end{center}
\vspace*{\fill}
\end{titlepage}

\clearpage
\thispagestyle{empty}
\noindent УДК 02 \\
ББК 78.34(2)7 \\
М 64

\vspace{8mm}

Миронов А. В. \textbf{Практика каталогизации в ИРБИС64} : методические материалы с уч\"eтом ГОСТ~7.0.100-2018 / А.~В.~Миронов. — Иркутск, 2019. — 44 с.

% TODO количество страниц!!!

\vspace{8mm}

Издание обобщает опыт эксплуатации АБИС "<ИРБИС64"> в Иркутской областной государственной универсальной научной библиотеке им. И. И. Мол\-ча\-но\-ва-Сибирского для автоматизации библиотечных процессов; содержит как общие рекомендации по вводу и подробные рекомендации по заполнению наиболее важных полей библиографического описания: сведений об индивидуальных авторах, заглавии, сведения об издании, выходные данные, количественные характеристики, страна издания, язык основного текста, ISBN, цена, общие примечания, примечания об интеллектуальной ответственности, область серии.

\vspace{3mm}
Издание рассчитано на каталогизаторов, имеющих базовые навыки работы с АРМ "<Каталогизатор">. Его цель – помощь в повышении квалификации каталогизаторов.

\vspace{5cm}

\begin{flushright}
\textcopyright Алексей Миронов, 2017-2019

\vspace{3mm}	

Иркутская областная государственная \\
универсальная научная библиотека \\
им. И. И. Молчанова-Сибирского, 2017-2019
\end{flushright}

\clearpage
\thispagestyle{empty}
\setcounter{tocdepth}{0}
\setlength\cftparskip{-7pt}
%\setlength\cftbeforesecskip{1pt}
%\setlength\cftaftertoctitleskip{2pt}
\tableofcontents

\mainmatter
\chapter*{Введение}

\setcounter{page}{5}

Съешь ещё этих мягких французских булок, да выпей, дружок, чаю. В чащах юга жил-был цитрус, но фальшивый экземпляр. Съешь ещё этих мягких французских булок, да выпей, дружок, чаю. В чащах юга жил-был цитрус, но фальшивый экземпляр. Съешь ещё этих мягких французских булок, да выпей, дружок, чаю. В чащах юга жил-был цитрус, но фальшивый экземпляр.

Съешь ещё этих мягких французских булок, да выпей, дружок, чаю. В чащах юга жил-был цитрус, но фальшивый экземпляр. Съешь ещё этих мягких французских булок, да выпей, дружок, чаю. В чащах юга жил-был цитрус, но фальшивый экземпляр. Съешь ещё этих мягких французских булок, да выпей, дружок, чаю. В чащах юга жил-был цитрус, но фальшивый экземпляр.

Съешь ещё этих мягких французских булок, да выпей, дружок, чаю. В чащах юга жил-был цитрус, но фальшивый экземпляр. Съешь ещё этих мягких французских булок, да выпей, дружок, чаю. В чащах юга жил-был цитрус, но фальшивый экземпляр. Съешь ещё этих мягких французских булок, да выпей, дружок, чаю. В чащах юга жил-был цитрус, но фальшивый экземпляр.



\chapter{ГОСТ 7.0.100-2018}

Согласно п. 4. 3 в состав библиографического описания входят следующие области в привед\"eнной ниже последовательности:

\begin{itemize}
    \item область заглавия и сведений об ответственности;
    \item область издания;
    \item специфическая область материала или вида ресурса;
    \item область публикации, производства, распространения и т. д.;
    \item область физической характеристики;
    \item область серии и многочастного монографического ресурса;
    \item область примечания;
    \item область идентификатора ресурса и условий доступности;
    \item область вида содержания и средства доступа.
\end{itemize}

Пункт 4. 4: области описания состоят из элементов, которые делятся на обязательные, условно обязательные и факультативные. В зависимости от набора элементов различают:

\begin{itemize}
    \item краткое библиографическое описание (содержит только обязательные элементы);
    \item расширенное библиографическое описание (содержит обязательные и условно-обязательные элементы);
    \item полное библиографическое описание (содержит обязательные, условно-обязательные и факультативные элементы).
\end{itemize}


\chapter{Общие рекомендации по заполнению полей}

\section{Пробельные символы и знаки препинания}

В элементах БО пробелы должны употребляться в соответствии со следующими требованиями:

\begin{itemize}
	\item Пробелы в начале и в конце элемента недопустимы;
	\item Два и более пробела подряд недопустимы (исключение составляют некоторые подполя полей 330 и 922);
	\item Пробел ставится после знаков препинания (точка, запятая, точка с запятой, вопросительный и восклицательный знаки), но не перед ними. Исключение составляет предписанная пунктуация, вводимая непосредственно в подполе (например, точка с запятой в последующих сведениях об ответственности обрамляется пробелами как спереди, так и сзади);
	\item Пробел ставится перед открывающей скобкой (круглой, квадратной), но не после неё;
	\item Пробел ставится после закрывающей скобки, но не до неё;
	\item Пробел не ставится между двумя соседствующими знаками препинания;
	\item Пробел ставится между числом и обозначением года/века или единицы измерения (например: \textit{2005 г., 100 м});
	\item Пробел ставится внутри инициалов (например: \emph{Миронов~А.~В.});
	\item Многоточие выделяется пробелом с обеих сторон (например: \emph{всё было б хорошо \dots }).
\end{itemize}

\section{Числа, числительные, римские цифры, единицы измерения}

Числительные во всех служебных полях БО всегда вводятся только в цифровой форме, а не в словесной (прописью).

При записи числительных в цифровой форме необходимо использовать арабские цифры, кроме явно оговор\"eнных случаев (например, наличие в описываемом документе двойной пагинации -- римской и арабской).

Числа записываются без пробелов (даже в тех случаях, когда в описываемом документе применяются пробелы для разбивки длинного числа на группы). Примеры:

\begin{itemize}
	\item \emph{20 000 лье под водой} $\rightarrow$ \emph{20000 лье под водой};
	\item \emph{8 201 794} $\rightarrow$ \emph{8201794}.
\end{itemize}

\textbf{Номер тома, выпуска, части} и~т.~п. всегда пишется только в цифровой форме в арабской нотации.

Правильно: \emph{Т. 1, Ч. 2, Вып. 3, Кн. 4.}

Неправильно: emph{Т II, Часть третья.}

\textbf{Тире в числовых интервалах.} Тире, обозначающее интервал значений, выделяют пробелами при словесной или смешанной форме чисел, например: \emph{пять -- семь минут}, \emph{20 тыс. -- 30 тыс.}, но приводят без пробелов при цифровой форме чисел, например: \emph{1941--1945}, \emph{с. 12--45}, \emph{разд. 6--11} (РПК, п. 4.4.3).

Поскольку символ тире отсутствует на клавиатуре, он заменяется дефисом (минусом).

\textbf{Римские цифры}. Любые числа, записанные в римской нотации (например, \emph{XXI}, \emph{MCMLXXIII} и~т. ~п.), должны вводиться строго лишь в латинской раскладке клавиатуры. Заменять латинские буквы \emph{X}, \emph{C} сходными по написанию кириллическими \emph{Х} и \emph{С} недопустимо! Также нельзя заменять букву \emph{I} на цифру \emph{1}. Вообще, при написании чисел в римской нотации допустимо использовать лишь прописные латинские буквы \emph{C}, \emph{I}, \emph{L}, \emph{M}, \emph{V} и \emph{X}.

В поле 10 "<Цена, ISBN"> последнюю (контрольную) цифру ISBN также вводят только в латинской раскладке.

\textbf{Единицы измерения величин}, как правило, вводятся без точки (ГОСТ 8.417-2002), однако, следует иметь в виду, что в обозначениях момента времени точка используется (Справочник издателя и автора, раздел 7.2, см. также ниже).

\smallskip
\noindent\begin{tabularx}{\linewidth}{|X|X|}
    \hline 
    \thead{Вводятся без точки} & \thead{Вводятся с точкой}  \\ 
    \hline 
    час (ч), минута (мин), секунда (сек) в обозначении интервала времени &  год (г.), месяц (мес.), час (ч.), минута (мин.) в обозначениях момента времени \\ 
    \hline 
    кило-, санти-, деци-, милли-, микро- метр (м), грамм (г), литр (л), градус (град)  & \\ 
    \hline 
    тонна (т), ценнер (ц)  &  \\ 
    \hline 
    ньютон (Н), ампер (А), ом (Ом), вольт (В), ватт (Вт) &  \\ 
    \hline 
\end{tabularx} 

\section{Форма написания дат и интервалов времени}

Написание дат и интервалов времени регулируется ГОСТ 7.64-90 "<Представление дат и времени дня. Общие требования">. Также о написании дат и интервалов времени см. "<Справочник издателя и автора">, глава 7, особенно раздел 7.2.

В элементах БО, таких как хронологические рубрики, подрубрики и аналогичных по назначению, обозначение единицы измерения интервала времени (года, века и т. д.) всегда отделяется от числа пробелом. Применяется сокращённая форма написания единицы с точкой на конце.

Правильно: \emph{2005 г., 1900-1950 гг., 19-20 вв.}

Неправильно: \emph{2005г., 1900-50гг., XIX в., XIX-XX вв.}

Правила написания интервалов времени см п. 4.4.3 РПК.

Правила написания дат и интервалов времени в элементах БО, таких как заглавие, см. в соответствующих разделах.

\section{Наращение окончаний числительных}

Наращение окончаний допустимо только для порядковых числительных, записанных арабскими цифрами. У количественных числительных окончание не наращивается никогда!

Наращение не используется:

\begin{cutelist}
    \item при записи календарных чисел: \emph{22 марта};
    \item если число обозначено римскими цифрами: \emph{IX конгресс};
    \item в номерах томов, глав, страниц, иллюстраций, таблиц, приложений, если родовое слово (том, глава) предшествует числительному: \emph{на с. 196}, \emph{в т. 5} (но \emph{на 196-й странице}, в \emph{5-м томе}).
\end{cutelist}

Наращение всегда пишется с дефисом (т. е. пробелы между числом и окончанием недопустимы). Правильно: \emph{2-е издание}. Неправильно: \emph{2~-е издание}.

Наращение падежного окончания в порядковых числительных, обозначенных арабскими цифрами, может быть однобуквенным или двухбуквенным.

По закрепившейся традиции наращение должно быть однобуквенным, если последней букве числительного предшествует гласный звук: \emph{5-й день} (пятый день), \emph{25-я годовщина} (двадцать пятая годовщина), \emph{в 32-м издании} (в тридцать втором издании), \emph{в 14-м} ряду (в четырнадцатом ряду).

Наращение должно быть двубуквенным, если последней букве предшествует согласный: \emph{5-го дня} (пятого дня), \emph{к 25-му студенту} (к двадцать пятому студенту), из \emph{32-го издания} (из тридцать второго издания), \emph{из 14-го ряда} (из четырнадцатого ряда).

Если подряд следуют два порядковых числительных, раздел\"eнных запятой или соедин\"eнных союзом, падежное окончание наращивают у каждого из них: \emph{1-й, 2-й вагоны}; \emph{80-е и 90-е годы}.

Если подряд следуют более двух порядковых числительных, раздел\"eнных запятой, точкой с запятой или соедин\"eнных союзом, то падежное окончание наращивают только у последнего числительного: \emph{1, 2 и 3-й вагоны}, \emph{70, 80, 90-е годы}.

Если два порядковых числительных следуют через тире, то падежное окончание наращивают:

\begin{itemize}
    \item[а)] только у второго числительного, если падежное окончание у обоих числительных одинаковое: \emph{50–60-е годы}, \emph{в 80–90-х годах};
    \item [б)] у каждого числительного, если падежные окончания разные: \emph{в 11-м – 20-х рядах}.
\end{itemize}

% Без наращения приводятся порядковые номера томов, глав, страниц, классов, курсов, если соответствующие родовые слова (том, глава и т. д.) предшествуют порядковому числительному.

\section{Кавычки и апострофы}

Во всех полях можно употреблять только машинописные двойные кавычки '', часто называемые "<лапками">, даже если в описываемом документе применяется другой тип кавычек.

Это связано с тем, что ИРБИС неверно обрабатывает любые другие кавычки. Апострофы ' и одинарные кавычки ‘ заменять на двойные кавычки не надо. Обратный апостроф ` в БО не используется.

% TODO Таблица

\section{Сокращение слов}

C 1 июля 2019 года вступил в силу ГОСТ Р 7.0.100-2018 "<Библиографическая запись. Библиографическое описание. Общие требования и правила составления">. В н\"eм значительно сокращается употребление сокращений в описаниях. 

% TODO далее

\section{Употребление букв Е и \"E}

Правила употребления буквы \"E приведены в академическом справочнике "<Правила русской орфографии и пунктуации"> под редакцией В.~В.~Лопатина. В частности, \S{} 5 утверждает следующее.

\begin{quotation}
    Употребление буквы \"e может быть последовательным и~выборочным.
    
    Последовательное употребление буквы \"e обязательно в~следующих разновидностях печатных текстов:
    
    а) в текстах с последовательно поставленными знаками ударения;
    
    б) в книгах, адресованных детям младшего возраста;
    
    в) в учебных текстах для школьников младших классов и иностранцев, изучающих русский язык.
    
    В обычных печатных текстах буква \"e употребляется выборочно. Рекомендуется употреблять ее в следующих случаях.
    
    1. Для предупреждения неправильного опознания слова, напримеп: \emph{вс\"e}, \emph{н\"eбо}, \emph{л\"eтом}, \emph{соверш\"eнный} (в отличие соответственно от слов \emph{все}, \emph{небо}, \emph{летом}, \emph{совершенный}), в том числе для указания на место ударения в слове, например: \emph{в\"eдро}, \emph{узна\"eм} (в отличие от \emph{ведро}, \emph{узнаем}).
    
    2. Для указания правильного произношения слова — либо редкого, недостаточно хорошо известного, либо имеющего распростран\"eнное неправильное произношение, например: \emph{г\"eзы}, \emph{с\"eрфинг}, \emph{фл\"eр}, \emph{тв\"eрже}, \emph{щ\"eлочка}, в том числе для указания правильного ударения, например: \emph{побас\"eнка}, \emph{привед\"eнный}, \emph{унес\"eнный}, \emph{осужд\"eнный}, \emph{новорожд\"eнный}, \emph{фил\"eр}.
    
    3. В собственных именах -- фамилиях, географических названиях, например: \emph{Кон\"eнков}, \emph{Не\"eлова}, \emph{Катрин Ден\"eв}, \emph{Шр\"eдингер}, \emph{Дежн\"eв}, \emph{Кошел\"eв}, \emph{Чебыш\"eв}, \emph{В\"eшенская}, \emph{Ол\"eкма}.
\end{quotation}

Таким образом, в электронном каталоге употребление буквы \"E выборочное. При заполнении полей всюду, где это возможно, букву \"E необходимо заменять на Е.
\chapter*{Системы личных имен у народов мира}

\section*{Испанские (иберийские) фамилии}

В испано- и португалоязычных регионах планеты существуют сходные правила построения имен, эти правила берут начало из ономастических традиций Испании и Португалии, известных под общим названием \textit{иберийское имя}.


\chapter{Поля 70x: индивидуальные авторы}

Блок полей 7xx в RUSMARC  -- блок ответственности, он содержит имена лиц и наименования организаций, надел\"eнных той или иной степенью ответственности по отношению к каталогизируемому документу (создание документа, его распространение, владение документом и т.п.).

Ответственность подразделяется на первичную и вторичную. Первичную ответственность могут нести одно лицо, одна организация или один род / династия / семья. Другие лица, роды / династии / семьи, организации, несущие равную с ними ответственность, наделены статусом альтернативной ответственности.

Если основной точкой доступа в записи является заглавие, лица, роды / династии / семьи и организации могут быть наделены статусом альтернативной или вторичной ответственности: авторы (один или несколько авторов с равной степенью ответственности) наделяются статусом альтернативной ответственности; редакторы, переводчики, авторы иллюстраций и т.д. -- статусом вторичной ответственности.

Если невозможно определить уровень ответственности, все лица и организации рассматриваются как несущие вторичную ответственность.

При внесении сведений в данное поле основным моментом является определение фамилии автора или первого элемента при ее отсутствии. Имя автора приводится в форме, получившей наибольшую известность. В качестве ориентира здесь должны выступать энциклопедические словари, справочники и национальный авторитетный файл.

\section{Подполе A: фамилия}

\section{Подполе B: инициалы}

\section{Подполе G: расширение инициалов}

\section{Подполе 9: роль (инвертирование ФИО допустимо?)}

Устанавливается в значение \emph{1} (соответствует положению переключателя \emph{Отменить умолчание}), если имя записано в прямом порядке.

Имя записано в прямом порядке:

\begin{itemize}
    \item ИМЯ;
    \item ИМЯ ОТЧЕСТВО;
    \item ИМЯ ПРОЗВИЩЕ (ЭПИТЕТ, ОПРЕДЕЛЕНИЕ);
    \item ИМЯ [ФАМИЛИЯ ИМЯ ОТЧЕСТВО] ДУХОВНОЕ ЗВАНИЕ;
    \item ИМЯ ИМЯ ИМЯ (китайские и другие восточные имена);
    \item СОКРАЩЕННОЕ ИМЯ (.) ИНИЦИАЛ;
    \item ИНИЦИАЛ (.) СОКРАЩЕННОЕ ИЛИ ПОЛНОЕ ИМЯ.
\end{itemize}

Имя записано под фамилией, родовым именем, отчеством:

\begin{itemize}
    \item ФАМИЛИЯ (,) ИМЯ ОТЧЕСТВО (то же в инициальной форме);
    \item ФАМИЛИЯ (,) ИМЯ (то же инициалы);
    \item ФАМИЛИЯ;
    \item ФАМИЛИЯ (-) ФАМИЛИЯ (,) ИМЯ ОТЧЕСТВО (то же в инициальной форме);
    \item ФАМИЛИЯ ФАМИЛИЯ ФАМИЛИЯ (,) ИМЯ (то же инициалы);
    \item СОКРАЩЕННАЯ ФАМИЛИЯ (,) ИМЯ (то же инициалы).
\end{itemize}

Если имя приведено в прямом порядке, то в подполях 9 и L должно быть установлено значение \emph{1}, а подполя B и G должны быть пустыми.

Заполнение подполя 9 влияет на автоматическое формирование сведений об ответственности. В частности, если установить его в \emph{1}, то в сформированных сведениях об ответственности инициалы будут помещены после фамилии автора. Поэтому, если на титульном листе ФИО автора приведено как \emph{Иванов А. А.}, то подполе 9 следует установить в \emph{1}.

\section{Подполе 1: неотъемлемая часть имени}

Неотъемлемая часть имени -- та, которая не может измениться в течение жизни. Если часть имени может измениться, то она не является неотъемлемой и заносится в подполе C.

\textbf{Примеры.}

\begin{itemize}
    \item барон;
    \item младший;
    \item отец;
    \item святитель;
    \item сын;
    \item Jr.
\end{itemize}

\textbf{Типичные ошибки.}

\begin{itemize}
    \item диакон;
    \item д-р, профессор Гейдельбергского университета;
    \item оглы;
    \item писатель, общественный деятель.
\end{itemize}

\section{Подполе C: дополнения к именам, кроме дат}

Любые дополнения к именам (кроме дат), которые не являются неотъемлемой частью имени (титулы, звания, эпитеты, указание должности).

Допускаются сокращения по ГОСТ.

\textbf{Примеры.}

\begin{itemize}
    \item доктор биологических наук, профессор;
    \item заслуж. учитель РСФСР, канд. пед. наук;
    \item чемпион мира по шахматам (1975-1985).
\end{itemize}

\textbf{Типичные ошибки.}

\begin{itemize}
    \item Димитр Христов Чорбаджийский;
    \item Епископ Диоклийский Каллист (Уэр);
    \item лорд;
    \item отец ; французский писатель.
\end{itemize}

% TODO Шаблоны дополнения

\section{Подполе L: индикатор формы записи имени}

\section{Подполе D: римские цифры}

\section{Подполе F: даты жизни}

\section{Подполе R: разночтение фамилии}

\section{Подполе Y: работает в данной организации}

\section{Подполе P: место работы автора}

\section{Подполя 4, 5, 6: функция}


\chapter{Поля 710x: коллективные авторы}


\chapter{Поле 200: основное заглавие}

Блок 2xx в RUSMARC -- блок описательной информации, он содержит данные для всех областей описания ISBD, кроме Области примечания и Области стандартного номера (или его альтернативы) и условий доступности стандартов ISBD и ГОСТ 7.1-2003. 

Согласно стандарту RUSMARC

\begin{quotation}
	Элементы данных в блоке 2{-}{-} не должны генерироваться автоматически из элементов данных полей других блоков исходной записи при отсутствии их в блоке. Например, если в исходной записи отсутствует элемент данных "<сведения об ответственности">, поля блока 7{-}{-} (сведения об ответственности в форме точки доступа) не должны использоваться для генерации этого элемента, так как нет гарантии, что данные в таком сгенерированном поле будут добавлять что-либо к описательной части записи.
\end{quotation}

Однако АРМ "<Каталогизатор"> нарушает это требование, и сведения об ответственности генерируются автоматически, если они не заполнены вручную.

Впрочем, это можно отключить в настройках АРМ.

Поле 200 в ИРБИС соответствует полю RUSMARC: 200

ГОСТ 7.1-2003, пункт 5.2.2.

Поле обязательное, не повторяющееся.

\section{Подполе A: основное заглавие}

Приводится полностью в том виде и в той последовательности, как оно дано в публикации, с сохранением имеющихся знаков препинания.

При отсутствии в заглавии знаков препинания между фразами они отделяются друг от друга точкой.

\textbf{Примеры:}

\begin{itemize}
	\item Инновации? Инновации… Инновации!
	\item Культура. Информация. Технология
	\item Десять лет, которые потрясли … 1991-2001
\end{itemize}

Основное заглавие может содержать альтернативное заглавие, соединенное с ним союзом "<или"> и записываемое с прописной буквы. Перед союзом «или» ставят запятую. Альтернативное заглавие повторяется в подполе 517\^{}а.

\textbf{Примеры:}

\begin{itemize}
	\item Come undone, или Макабрические миры доктора Делюмо
	\item 47 принципов древних самураев, или Кодекс руководителя
	\item Августейший сезон, или Книга российских календ
\end{itemize}

При наличии в начале заглавия инициалов или раскрытых инициалов с фамилией лица, то заглавие без инициалов или раскрытых инициалов отражается дополнительно в подполе 517\^{}а.

\textbf{Примеры:}

\begin{itemize}
	\item Владимир Владимирович Маяковский -- человек и поэт
	\item А. С. Макаренко о воспитании детей в семье
	\item Александр Вампилов в воспоминаниях и фотографиях
\end{itemize}

Если основное заглавие начинается с цифрового обозначения, то в подполе 517\^{}а повторяются варианты заглавия:

\begin{itemize}
	\item[а)] цифра дается в словесном выражении и наоборот;
	\item[б)] римская цифра заменяется арабской и наоборот;
	\item[в)] без цифрового обозначения (для чтений, посвященных лицу).
\end{itemize}

\textbf{Пример:}

\begin{itemize}
	\item 200\^{}а – Х Международная научная конференция документоведов
	\item 517\^{}а – 10 Международная научная конференция документоведов
	\item 517\^{}а – Десятая Международная научная конференция документоведов
\end{itemize}

Если заглавие состоит из двух и более значимых фраз, то в поле 517 создается точка доступа на все фразы, кроме первой, при этом тип разночтения устанавливается "<517 Вариант заглавия">. В подполе "<Роль (основная карточка)"> ставится переключатель "<Нет">.

\textbf{Примеры.}

\begin{itemize}
	\item Автоматическое регулирование. Теория и элементы управляющих систем
	\item Беседы о русской культуре. Быт и традиции русского дворянства
	\item Введение в изучение права и нравственности. Эмоциональная психология 
\end{itemize}

Если в предписанном источнике информации типовые (нехарактерные) заглавия приведены через точку, например, "<Романы. Повести. Рассказы"> или в отдельных строках, такое заглавие вводится целиком в 200\^{}a плюс создается дополнительная точка доступа на каждое заглавие, стоящее после точки ("<Повести"> и "<Рассказы"> в вышеприведенном примере).

Если часть заглавия не представляет собой самостоятельно значимой фразы, она не повторяется в поле 517.

\textbf{Примеры.}

\begin{itemize}
	\item Философия современной России. Какая она?
	\item Мы не пойдем на Север. Почему?
\end{itemize}

При отсутствии в публикации заглавия, его формулируют и приводят в квадратных скобках.

Недопустима формулировка "<Без заглавия">.

В конце заглавия точка не ставится, кроме случаев, когда оно заканчивается многоточием.

В заглавии сокращения слов не допускаются, кроме случаев, когда слова сокращены автором.

% TODO орфография и далее

\section{Подполе B: общее обозначение материала}

{\color{red}В новом ГОСТ больше не используется}

ГОСТ 7.1-2003, пункт 5.2.3

Указывает на носитель информации, содержащего каталогизируемый документ.

\begin{tabular}{|l|l|}
	\hline 
	\thead{На русском языке}& \thead{На иностранном языке}  \\ 
	\hline 
	Видеозапись &  Videorecording \\ 
	\hline 
	Звукозапись &  Sound recording \\ 
	\hline 
	Изоматериал &  Graphic \\ 
	\hline 
	Карты &  Cartographic material \\ 
	\hline 
	Комплект &  Kit \\ 
	\hline 
	Кинофильм &  Motion picture \\ 
	\hline 
	Микроформа &  Microform \\ 
	\hline 
	Мультимедиа &  Multimedia \\ 
	\hline 
	Ноты &  Music \\ 
	\hline 
	Предмет &  Object \\ 
	\hline 
	Рукопись &  Manuscript \\ 
	\hline 
	Текст &  Text \\ 
	\hline 
	Шрифт Брайля &  Braille \\ 
	\hline 
	Электронный ресурс & Electronic resource \\ 
	\hline 
\end{tabular} 

% TODO мультимедиа и проч

\section{Подполе E: сведения, относящиеся к заглавию}

ГОСТ 7.1-2003, пункт 5.2.5

Сведения в подполе раскрывают и поясняют заглавие произведения. Здесь также приводятся сведения о литературном жанре произведения, указание о том, что оно является переводом с другого языка (при отсутствии фамилии переводчика) и т. п.

Сведения указываются в том порядке и в том виде, как они приведены в публикации. Если сведений в предписанном источнике информации нет, их можно взять из другого источника или сформулировать самостоятельно. В этом случае сведения приводятся в квадратных скобках. Типичные сведения для печатных изданий:

% TODO альманах и т. д.

\section{Подполя F: первые сведения об ответственности и G: последующие сведения об ответственности}

ГОСТ 7.1-2003, пункт 5.2.6

Сведения об ответственности содержат информацию о лицах и организациях, участвовавших в создании интеллектуального, художественного или иного содержания произведения, являющегося объектом описания. Сведения об ответственности записывают в той форме, в какой они указаны в предписанном источнике информации.

Сведения приводятся со строчной буквы, кроме случаев, когда они начинаются с имен собственных или аббревиатуры, начинающейся с прописной буквы.

Сокращение отдельных слов и словосочетаний производится согласно ГОСТ.
Для произведений, имеющих одного, двух и трех авторов, приводятся сведения об этих авторах в том виде, как они указаны на титульном листе.

Для произведений, имеющих четырех и более авторов, приводятся сведения о первом авторе в том виде, как он указан в статье, с добавлением в квадратных скобках слов [и др.].

Если сведения об авторе являются составной частью заглавия, то в первых сведениях об ответственности они не приводятся. Примеры: «Дневник А. И. Деникина», «Письма А. А. Ахматовой».

Никогда не отражаются в подполе 200\^{}f авторы произведений из подборок, рецензируемых книг, статей и авторы выступлений на мероприятиях. Также никогда не отражаются авторы цитат.

В ИРБИС предусмотрено три способа заполнения подполя:

\begin{itemize}
	\item простой ввод с клавиатуры -- самый гибкий, но одновременно и самый трудоёмкий способ;
	\item с использованием словаря авторов, когда система подставляет в подполе список отобранных авторов, разделённых запятыми, если необходимо, автоматически выполняя инверсию ФИО. Временные и постоянные коллективные авторы разделяются точкой с запятой;
	\item если подполе не заполнено на момент сохранения записи, оно автоматически формируется из введенных сведений об авторах и лицах со вторичной ответственностью. При этом, если необходимо, выполняется инверсия ФИО. Автоматическое формирование подполя можно отключить в настройках АРМ, если в настройке (по кнопке или в поле 905 на странице Технология) задать "<нет"> для признака "<Формировать Сведения об ответственности?"> (по умолчанию сведения об ответственности формируются автоматически).
\end{itemize}

% TODO предпочтительный порядок

\section{Подполе U: роль (нехарактерное заглавие, доб. карточка?)}

Согласно "<Издательскому словарю">

\begin{quotation}
	ХАРАКТЕРНОЕ ЗАГЛАВИЕ -- вид тематического заглавия, словесно выражающего инд. тему, идею, образ, цель, адрес издания, в отличие от заглавия типового (нехарактерного).
	
	ТИПОВОЕ ЗАГЛАВИЕ -- тематическое заглавие издания, определяющее лишь его жанр, вид, назначение и в ряде случаев неотличимое от таких же заглавий других изданий. Напр.: Путеводитель, Каталог, Бюллетень, Труды, Науч. труды.	
\end{quotation}

% TODO кроме того

\section{Подполе V: обозначение и номер тома}

Подполе вводится в одном из двух форматов.

1. Сокращ\"eнное обозначение части многотомного издания, пробел и порядковый номер части.

{\noindent\scriptsize\begin{tabular}{|l|l|l|l|}
	\hline 
	\thead{Обозначение части} & \thead{Сокращение}  & \thead{Обозначение части}  & \thead{Сокращение}  \\ 
	\hline 
	\multicolumn{2}{|c|}{для русскоязычных документов} &  \multicolumn{2}{|c|}{для документов на иностранных языках}  \\ 
	\hline 
	Альбом & Альб. & Album & Alb.  \\ 
	\hline 
	Выпуск &  Вып. &  Issue & Iss.  \\ 
	\hline 
	Диск &  Диск &  Disk & Disk  \\ 
	\hline 
	Дополнительный выпуск &  Доп. вып. & Additional issue & Add. iss.  \\ 
	\hline 
	Дополнительный том &  Доп. т. &  Additional volume & Add. vol.  \\ 
	\hline 
	Кассета & Кас. &  Cassette & Cas. \\ 
	\hline 
	Книга & Кн. & Book & Book \\ 
	\hline 
	Сборник &  Сб. & Collection & Col.  \\ 
	\hline 
	Спецвыпуск & Спецвып. & Special issue & Spec. iss. \\ 
	\hline 
	Специальный выпуск &  Спец. вып. & Special issue & Spec. iss. \\ 
	\hline 
	Том &  Т. & Volume & Vol. \\ 
	\hline 
	Часть &  Ч. & Part  & Pt.  \\ 
	\hline 
\end{tabular} 
}
\chapter{Поле 205: сведения об издании}

ГОСТ 7.1-2003, п. 5.3

Поле повторяется (например, в случае ошибочного указания издания).

Содержит информацию об изменениях и особенностях данного издания по отношению к предыдущему изданию того же произведения.

\section*{Подполе A: сведения об издании}

Содержит: сведения (слово, фраза или группа символов), имеющиеся в пред-писанном источнике информации, либо сформулированные каталогизатором в соответствии с Правилами, и идентифицирующие документ с точки зрения издания.

Следует использовать встроенный справочник \textit{205.mnu}.

Номер издания следует приводить в цифровой форме в арабской нотации с наращением окончания числительного, независимо от того, в какой форме оно приведено в предписанном источнике информации.

\chapter{Поле 210: выходные данные}

Соответствует полю RUSMARC: 210

ГОСТ 7.1-2003, п. 5.5

Повторяется для каждого издательства.

ГОД ИЗДАНИЯ (подполе d) — последний элемент выходных данных издания. Им по закону РФ "<Об авторском праве и смежных правах"> от 09.07.1993 N 5351-1 считается год выпуска в обращение экземпляров произведения (издания), т. е. год сдачи тиража или его начальной партии книготорговцу. Указывается арабскими цифрами (без сокращенного или полного слова "<год">. ГОСТ 7.4—95 требует в повторных изданиях проставлять на обороте титульном листе год выпуска предшествующего издания (напр.: 2-е издание вышло в 1975 г.), а в многотомных -- год выпуска первого тома, т. е. начала выпуска всего многотомного издания (напр.: Т. 1 вышел в 1991 г.), однако при контртитуле с общими для всего многотомного издания выходными сведениями целесообразно в выходных данных ставить год выпуска 1-го тома с висячим тире во всех томах, кроме последнего, где указывают год выпуска первого тома и через тире -- последнего.

В периодических (кроме газет) и продолжающихся изданиях год издания указывают при номере издания независимо от года выхода его в свет.

\textbf{Примеры.} \textbf{Один город, два издательства}

\begin{tabular}{| l | l | l |}
	\hline
	\thead{Поле} & \thead{Подполе} & \thead{Значение} \\
	\hline
	\multicolumn{3}{|l|}{text210: Выходные данные} \\
	\hline
	& a: Город1 & Москва \\
	\hline
	& c: Издательство & ИНФРА-М \\
	\hline
	& d: Год издания & 2013 \\
	\hline
	\multicolumn{3}{|l|}{210: Выходные данные -- повторение} \\
	\hline
	& a: Город1 & Москва \\
	\hline
	& c: Издательство & Новое знание \\
	\hline
	& ?: Роль (Города не выводить?) & 1 \\
	\hline
\end{tabular}

\smallskip
Результат
\smallskip

\noindent\fbox{
	\begin{minipage}{\linewidth}
		Москва : ИНФРА-М : Новое знание, 2013
	\end{minipage}
}
\smallskip

\textbf{Два города, два издательства}

\begin{tabular}{| l | l | l |}
	\hline
	\thead{Поле} & \thead{Подполе} & \thead{Значение} \\
	\hline
	\multicolumn{3}{|l|}{210: Выходные данные} \\	
	\hline
	& a: Город1 & Москва \\
	\hline
	& x: Город2 & Санкт-Петербург \\
	\hline
	& c: Издательство & ИНФРА-М \\
	\hline
	& d: Год издания & 2013 \\
	\hline
	\multicolumn{3}{|l|}{210: Выходные данные -- повторение} \\	
	\hline
	& a: Город1 & Москва \\
	\hline
	& x: Город2 & Санкт-Петербург \\
	\hline
	& c: Издательство & Новое знание \\
	\hline
	& ?: Роль (Города не выводить?) & 1 \\
	\hline
\end{tabular}

\smallskip
Результат
\smallskip

\noindent\fbox{
	\begin{minipage}{\linewidth}
		Москва ; Санкт-Петербург : ИНФРА-М : Новое знание, 2013
	\end{minipage}
}
\smallskip

\textbf{Два издательства в двух разных городах}

\begin{tabular}{| l | l | l |}
	\hline
	\thead{Поле} & \thead{Подполе} & \thead{Значение} \\
	\hline
	\multicolumn{3}{|l|}{210: Выходные данные} \\	
	\hline
	& a: Город1 & Москва \\
	\hline
	& c: Издательство & ИНФРА-М \\
	\hline
	& d: Год издания & 2013 \\
	\hline
	\multicolumn{3}{|l|}{210: Выходные данные -- повторение} \\	
	\hline
	& a: Город1 & Санкт-Петербург \\
	\hline
	& c: Издательство & Новое знание \\
	\hline
\end{tabular}

\smallskip
Результат
\smallskip

\noindent\fbox{
	\begin{minipage}{\linewidth}
		Москва : ИНФРА-М ; Санкт-Петербург : Новое знание, 2013
	\end{minipage}
}

\chapter{Поле 215: количественные характеристики}

ГОСТ 7.1-2013, п. 5.6 % TODO Ссылка на новый ГОСТ

Повторяется для разных физических носителей.

Если в подполе "<Объ\"eм"> 215\^{}a используется двойная (тройная) пагинация, надо использовать справочник-меню.

Подполе "<Единица измерения"> 215\^{}1 не должно содержать \emph{с.} или \emph{л.}.

\section{Подполе A: объ\"eм (цифры)}

\textbf{Примеры:}

\begin{itemize}
    \item 78
    \item XVI, 680
    \item [22], 290
    \item [12], XVII, 518
\end{itemize}

\textbf{Типичные ошибки:}

\begin{itemize}
    \item 201-415
    \item XLVI
\end{itemize}



\chapter{Поле 101: язык основного текста}

Поле обязательное.

Повторяемое. Повторяется для каждого языка, используемого в тексте.

Коды языков всегда вводятся строчными латинскими буквами, состоят из трех символов и берутся строго из встроенного справочника \emph{jz.mnu}!

Первым указывается язык, применяемый в данном документе в наибольшей степени. При неоднозначности (непонятно, какой из двух языков основной для документа) рекомендуется использовать код русского языка \emph{rus}, если, конечно, в документе имеется русский язык.

Небольшие цитаты (объёмом не более одного абзаца) не учитываются!

Для многоязычных документов (более 3 языков) можно использовать код языка \emph{mul} – "<мультиязычный документ">.

Для документов, язык которого определить невозможно (в том числе по причине отсутствия письменного текста и/или аудиосопровождения), нужно ис-пользовать код \emph{und} – <"не определено">.

Нельзя использовать вместо трехбуквенного кода языка двухбуквенный код страны.

\chapter*{Поле 102: страна}

Поле обязательное.

Повторяемое. Повторяется для каждой из стран, в которой издавался документ.

Перечень названий стран и их коды по ГОСТ 7.67-2003, введенному в действие 1 января 2005 года.

Постановлением Госстандарта РФ от 14 декабря 2001 г. N 529-ст)(с изменениями N 1/2003, 2/2003, 3/2004, 4/2004, 5/2005) в 2001 г. при-нят введён в действие Общероссийский классификатор стран мира OK (MK (ИСО 3166) 004-97) 025—2001 (ОКСМ).

Коды стран всегда вводятся прописными латинскими буквами, состоят из двух символов и берутся строго из встроенного справочника str.mnu!

Первой указывается… % TODO дописать

Заполняется строго значениями из справочника!

Если в справочнике отсутствует код страны, в которой был издан документ, то используется код страны, исторически последующей.

\chapter{Поле 10: ISBN, цена}

ГОСТ 7.1-2003, п. 5.9 % TODO Ссылка на новый ГОСТ

Поле повторяется для нескольких ISBN одного издания.
 
Вопросы, связанные с Международным стандартным книжным номером (ISBN) регулируются ГОСТ 7.0.53-2007.

ISBN действующего стандарта состоит из тринадцати цифр, поделенных на пять групп, разделяемых знаком дефиса.

\begin{enumerate}
    \item Префикс EAN.UCC -- код \emph{978} (в дальнейшем будет использоваться \emph{979}), предоставленный Европейской ассоциацией товарной нумерации (EAN) Международному агентству ISBN для обозначения товара "<Книжная продукция">.
    \item Номер регистрационной группы служит для обозначения в ISBN страны, географической или языковой области. Для Российской Федерации номер регистрационной группы -- цифра \emph{5}.
    \item Номер регистранта идентифицирует в системе ISBN конкретного издателя, производителя документов. Номер регистранта российский издатель (производитель документов) получает в Российском национальном агентстве ISBN, функционирующем в составе Российской книжной палаты.
    \item Номер издания (публикации) идентифицирует конкретное издание (публикацию) издателя, производителя документов в предоставленном ISBN.
    \item Контрольная цифра служит для проверки правильности цифровой части ISBN.
\end{enumerate}

ISBN старого стандарта состоял из десяти цифр, поделенных на четыре группы, -- в нем не было префикса EAN, остальные группы совпадают с действующим стандартом.

\textbf{Примеры.}

\begin{itemize}
	\item 0-812-57558-X \textit{(старый стандарт)};
	\item 5-02-003157-7;
	\item 978-5-98846-049-7 \textit{(действующий стандарт)}.
\end{itemize}

В издании, выпущенном совместно несколькими издателями (в том числе, российскими и зарубежными издателями) приводят ISBN каждого издателя-партнера. Наименование издателя указывают после соответствующего ISBN в круглых скобках без кавычек в той форме, как оно приведено на титульной странице.

\textbf{Примеры.}

\begin{itemize}
	\item 978-5-09-014485-8 (Просвещение);
	\item 978-5-472-01012-2 (Экзамен);
	\item 978-1 -84334-151-2 (Chandos Publishing);
	\item 978-5-93913-059-3 (Профессия).
\end{itemize}

В томе (выпуске) многотомного издания приводят ISBN данного тома (с указанием в круглых скобках обозначения и номера тома) и ISBN многотомного издания в целом.

\textbf{Пример.}

\begin{itemize}
	\item ISBN 978-5-02-033899-9 (т. 1);
	\item ISBN 978-5-02-033897-5.
\end{itemize}

В издании, входящем в состав комплектного, комбинированного издания, приводят ISBN данного издания (с указанием в круглых скобках сведений "<отд. кн."> или "<отд. изд.">) и ISBN комплектного, комбинированного издания в целом.

\textbf{Пример.}

\begin{itemize}
	\item 978-5-89349-822-6 (отд. кн.)
	\item 978-5-89349-820-2
\end{itemize}

ISBN комплектного комбинированного издания в целом приводят на футляре, папке, обложке комплектного издания.

В поле вводятся только ISBN, относящиеся непосредственно к обрабатываемому документу. Прочие ISBN, относящиеся, например, к электронной версии или к изданию в твердом переплете, опускаются. Также опускается ISBN, относящийся к многочастному документу в целом.
Если невозможно определить, какой ISBN к чему относится, то приводят все имеющиеся ISBN (ГОСТ 7.1-2003, п. 5.9.2). % TODO ссылка на новый ГОСТ

\section{Подполе A: ISBN}

Подполе имеет ФЛК, проверяющее правильность написания ISBN по контрольной цифре и сверяющее его на дублетность. Контроль не блокирующий, при наличии ошибок можно продолжать ввод и сохранять запись.

\textbf{Типичные ошибки.}

\begin{itemize}
    \item употребление кириллической буквы Х вместо латинской X;
    \item употребление пробелов вместо дефисов в качестве разделителя групп;
    \item запись ISBN без разделителей;
    \item ввод уточняющих сведений, которые должны быть внесены в подполе B.
\end{itemize}

\section{Подполе B: уточнения}

\textbf{Примеры.}

\begin{itemize}
    \item Бином
    \item Кн. 1
    \item в пер.
\end{itemize}

Типичная ошибка: ввод сведений в скобках.

\section{Подполе Z: ошибочный ISBN}

Подполе Z отличается от подполя A отсутствием ФЛК. Сюда нужно вносить ISBN, про которые точно известно, что они ошибочные.

\section{Подполе D: общая для всех экземпляров цена}

\textbf{Примеры.}

\begin{itemize}
    \item 1.00
    \item 1000.00
    \item 123.45
\end{itemize}

\textbf{Типичные ошибки.}

\begin{itemize}
    \item 5 руб.
    \item 10 коп.
    \item 10,50
    \item 12
\end{itemize}

\section{Подполе C: обозначение валюты}

\textbf{Примеры.}

\begin{itemize}
    \item Fr
    \item USD
    \item тенге
\end{itemize}

\textbf{Типичные ошибки.}

\begin{itemize}
    \item руб.
    \item \$
\end{itemize}

\chapter{Поля 461 и 46: общие сведения для многотомника}

Съешь ещё этих мягких французских булок, да выпей, дружок, чаю. В чащах юга жил-был цитрус, но фальшивый экземпляр. Съешь ещё этих мягких французских булок, да выпей, дружок, чаю. В чащах юга жил-был цитрус, но фальшивый экземпляр. Съешь ещё этих мягких французских булок, да выпей, дружок, чаю. В чащах юга жил-был цитрус, но фальшивый экземпляр.

Съешь ещё этих мягких французских булок, да выпей, дружок, чаю. В чащах юга жил-был цитрус, но фальшивый экземпляр. Съешь ещё этих мягких французских булок, да выпей, дружок, чаю. В чащах юга жил-был цитрус, но фальшивый экземпляр. Съешь ещё этих мягких французских булок, да выпей, дружок, чаю. В чащах юга жил-был цитрус, но фальшивый экземпляр.

Съешь ещё этих мягких французских булок, да выпей, дружок, чаю. В чащах юга жил-был цитрус, но фальшивый экземпляр. Съешь ещё этих мягких французских булок, да выпей, дружок, чаю. В чащах юга жил-был цитрус, но фальшивый экземпляр. Съешь ещё этих мягких французских булок, да выпей, дружок, чаю. В чащах юга жил-был цитрус, но фальшивый экземпляр.
\chapter*{Поле 300: общие примечания}

Соответствует полю RUSMARC: 300

ГОСТ 7.1-2003, п. 5.8

Повторяется. Если вводится более одного примечания, каждое вводится в новое повторение поля.

Содержит дополнительную информацию об объекте описания, которая не была приведена в других элементах описания. Сведения, приводимые в поле, заимствуют из любого источника и в квадратные скобки не заключают.

Слова текста, заключенные в угловые скобки < >,  включаются в словарь ключевых слов.

\textbf{Примеры.}

\begin{itemize}
	\item В надзаг.: Посвящается 60-летию Уфимского государственного нефтяного технического университета
	\item Деп. в ВИНИТИ 18.05.02, № 14432
	\item Издание выходило полутомами с последовательной нумерацией выпусков на корешках переплета
	\item Лауреат конкурса «Профессиональный учебник»
	\item На корешке указан том серии
	\item На шмуцтитуле: «Издано под наблюдением Комиссии при Комитете состоящего под высочайшим государя императора покровительством Императорского Общества любителей древней письменности»
	\item Посвящается памяти академика Д. С. Белянкина
	\item Продолжение романа «Ветер над полем»
	\item Произведение печатается без сокращений
	\item Электронные версии книг на сайте www.prospeckt.org
\end{itemize}

\textbf{Типичные ошибки.}

\begin{itemize}
	\item (Библиотека детского романа)
	\item (На обл: Опыт передового учителя)
	\item Авторы указаны на обороте тит. л. - Библиогр.: с. 287 (9 назв.).
	\item Загл. корешка: Великий Октябрь и реальный социализм
	\item Часть текста: англ.
\end{itemize}


\chapter{Поле 314: примечания об интеллектуальной ответственности}

Не повторяется.

\textbf{Примеры.}

\begin{itemize}
	\item Авторы указаны в оглавлении;
	\item Авторы указаны на обороте титульного листа;
	\item Заглавие с этикетки видеодиска;
	\item Сведения об авторах: с. 693-698.
\end{itemize}

\textbf{Типичные ошибки.}

\begin{itemize}
	\item Книга фактически издана в 2012 г.;
	\item Указ. имен и ил.: с. 415-424;
	\item Писательница Серова Марина -- это не реальный человек, а многочисленный коллектив авторов из Саратова, работающих под руководством одного литературного агента;
	\item Ранее кн. выходила под назв. «Адская смесь».
\end{itemize}

\chapter{Поле 225: область серии}

ГОСТ 7.1-2003, п. 5.7

Поле содержит заглавие серии вместе с любыми сведениями, относящимися к этому заглавию, и сведениями об ответственности, относящимися к заглавию серии.

Повторяется, если документ входит более чем в одну серию.

\section{Подполе A: Наименование}

Заглавие серии в той форме, в которой оно представлено в источнике информации.

Подполе обязательное.

\textbf{Примеры.}

\begin{itemize}
	\item Библиотечка автомобилиста;
	\item За Родину! За Победу!
	\item Классики мировой психологии
\end{itemize}

\textbf{Типичные ошибки.}

\begin{itemize}
	\item Серия "Высшее образование"
\end{itemize}

\section{Подполе V: Обозначение и Номер выпуска в серии}

Номер каталогизируемого документа внутри серии.

\textbf{Примеры.}

\begin{itemize}
	\item Вып. 22
	\item Кн. 6
	\item Т. 40	
\end{itemize}

\textbf{Типичные ошибки.}

\begin{itemize}
	\item Вып. № 22
	\item Малая серия
	\item Т. CXXVII
\end{itemize}

\include{conclusion}

\backmatter

\chapter{Примеры библиографических описаний}

Съешь ещё этих мягких французских булок, да выпей, дружок, чаю. В чащах юга жил-был цитрус, но фальшивый экземпляр. Съешь ещё этих мягких французских булок, да выпей, дружок, чаю. В чащах юга жил-был цитрус, но фальшивый экземпляр. Съешь ещё этих мягких французских булок, да выпей, дружок, чаю. В чащах юга жил-был цитрус, но фальшивый экземпляр.

Съешь ещё этих мягких французских булок, да выпей, дружок, чаю. В чащах юга жил-был цитрус, но фальшивый экземпляр. Съешь ещё этих мягких французских булок, да выпей, дружок, чаю. В чащах юга жил-был цитрус, но фальшивый экземпляр. Съешь ещё этих мягких французских булок, да выпей, дружок, чаю. В чащах юга жил-был цитрус, но фальшивый экземпляр.

Съешь ещё этих мягких французских булок, да выпей, дружок, чаю. В чащах юга жил-был цитрус, но фальшивый экземпляр. Съешь ещё этих мягких французских булок, да выпей, дружок, чаю. В чащах юга жил-был цитрус, но фальшивый экземпляр. Съешь ещё этих мягких французских булок, да выпей, дружок, чаю. В чащах юга жил-был цитрус, но фальшивый экземпляр.
\chapter{Клавиатурные сокращения}

\begin{tabularx}{\linewidth}{| X | l |}
	\hline
	\textbf{Действие} & \textbf{Клавиши} \\
	\hline
	Восстановление исходного значения поля, отказ от редактирования & Esc \\
	\hline
	Переход к следующему полю или подполю & Enter или $\downarrow$ \\
	\hline
	Новое повторение поля & Ctrl + $\downarrow$ \\
	\hline
	Переключение между вкладками рабочей области & Ctrl + $\rightarrow$ или $\leftarrow$ \\
	\hline
	Переход в конец страницы ввода & Page Down \\
	\hline
	Переход в начало страницы ввода & Page Up \\
	\hline
	Вызов вложенного рабочего листа, словаря или средства ввода & F2 \\
	\hline
	Мультиввод (повторения поля в виде таблицы) & F3 \\
	\hline
	Оперативное меню: сокращения по ГОСТ 7.12-93 и ГОСТ Р 7.0.12-2011, римские цифры, коды языков, команды контекстного выделения & F4 \\
	\hline
	Исправление кириллических символов, ошибочно набранных латиницей & F6 \\
	\hline
	Исправление символов, ошибочно введ\"eнных в верхнем регистре (с включенным Caps Lock) & F7 \\
	\hline
	Сохранение записи & Shift + Enter \\
	\hline
	Создание новой записи & Alt + Плюс на цифровой клавиатуре \\
	\hline
	Ввод текущей даты & Alt + Д \\
	\hline	
	Переход к первой записи & Shift + Page Up \\
	\hline
	Переход к последней записи & Shift + Page Down \\
	\hline
	Переход к следующей записи & Shift + $\downarrow$ \\
	\hline
	Переход к предыдущей записи & Shift + $\uparrow$ \\
	\hline
	Выход из вложенного рабочего листа, словаря или средства ввода с сохранением ввода & Tab, затем Enter \\
	\hline
	Выход из вложенного рабочего листа, словаря или средства ввода с отменой внесенных изменений & Последовательность Tab-Tab-Enter \\
	\hline
	Виртуальная клавиатура & Alt + V \\
	\hline
	Виртуальная клавиатура языковая & Alt + K (латинская) \\
	\hline
	Импорт из ЛИБНЕТ & Alt + I \\
	\hline
	Импорт из ИРБИС-корпорации & Alt + W \\
	\hline
	Импорт из Z-ресурсов & Alt + Z \\
	\hline
\end{tabularx}



% bibliography, glossary and index would go here.

\end{document}