\chapter{Клавиатурные сокращения}

\begin{tabularx}{\linewidth}{| X | l |}
	\hline
	\textbf{Действие} & \textbf{Клавиши} \\
	\hline
	Восстановление исходного значения поля, отказ от редактирования & Esc \\
	\hline
	Переход к следующему полю или подполю & Enter или $\downarrow$ \\
	\hline
	Новое повторение поля & Ctrl + $\downarrow$ \\
	\hline
	Переключение между вкладками рабочей области & Ctrl + $\rightarrow$ или $\leftarrow$ \\
	\hline
	Переход в конец страницы ввода & Page Down \\
	\hline
	Переход в начало страницы ввода & Page Up \\
	\hline
	Вызов вложенного рабочего листа, словаря или средства ввода & F2 \\
	\hline
	Мультиввод (повторения поля в виде таблицы) & F3 \\
	\hline
	Оперативное меню: сокращения по ГОСТ 7.12-93 и ГОСТ Р 7.0.12-2011, римские цифры, коды языков, команды контекстного выделения & F4 \\
	\hline
	Исправление кириллических символов, ошибочно набранных латиницей & F6 \\
	\hline
	Исправление символов, ошибочно введ\"eнных в верхнем регистре (с включенным Caps Lock) & F7 \\
	\hline
	Сохранение записи & Shift + Enter \\
	\hline
	Создание новой записи & Alt + Плюс на цифровой клавиатуре \\
	\hline
	Ввод текущей даты & Alt + Д \\
	\hline	
	Переход к первой записи & Shift + Page Up \\
	\hline
	Переход к последней записи & Shift + Page Down \\
	\hline
	Переход к следующей записи & Shift + $\downarrow$ \\
	\hline
	Переход к предыдущей записи & Shift + $\uparrow$ \\
	\hline
	Выход из вложенного рабочего листа, словаря или средства ввода с сохранением ввода & Tab, затем Enter \\
	\hline
	Выход из вложенного рабочего листа, словаря или средства ввода с отменой внесенных изменений & Последовательность Tab-Tab-Enter \\
	\hline
	Виртуальная клавиатура & Alt + V \\
	\hline
	Виртуальная клавиатура языковая & Alt + K (латинская) \\
	\hline
	Импорт из ЛИБНЕТ & Alt + I \\
	\hline
	Импорт из ИРБИС-корпорации & Alt + W \\
	\hline
	Импорт из Z-ресурсов & Alt + Z \\
	\hline
\end{tabularx}

