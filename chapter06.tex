\chapter*{Поле 210: выходные данные}

Соответствует полю RUSMARC: 210

ГОСТ 7.1-2003, п. 5.5

Повторяется для каждого издательства.

ГОД ИЗДАНИЯ (подполе d) — последний элемент выходных данных издания. Им по закону РФ "Об авторском праве и смежных правах" от 09.07.1993 N 5351-1 считается год выпуска в обращение экземпляров произведения (издания), т. е. год сдачи тиража или его начальной партии книготорговцу. Указывается арабскими цифрами (без сокращенного или полного слова "год". ГОСТ 7.4—95 требует в повторных изданиях проставлять на обороте титульном листе год выпуска предшествующего издания (напр.: 2-е издание вышло в 1975 г.), а в многотомных -- год выпуска первого тома, т. е. начала выпуска всего многотомного издания (напр.: Т. 1 вышел в 1991 г.), однако при контртитуле с общими для всего многотомного издания выходными сведениями целесообразно в выходных данных ставить год выпуска 1-го тома с висячим тире во всех томах, кроме последнего, где указывают год выпуска первого тома и через тире -- последнего.

В периодических (кроме газет) и продолжающихся изданиях год издания указывают при номере издания независимо от года выхода его в свет.
